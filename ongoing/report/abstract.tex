
\starttitle [title={Abstract}]

Multiple-input-multiple-output (MIMO) communication is widely expected to be the next-generation wireless communication scheme.
However, reliable communication in such r\`egime usually requires that the channel response be known at the receiver, necessary for example in beamforming algorithm and channel calibration.
It is then a non-trivial problem to estimate real-world wireless channels.

Meanwhile, in the millimeter wave r\`egime, which is usually used in tandem with MIMO, channel often exhibits sparse properties.
If the sparsity is exploited, few observations may suffice to estimate the channel, and such algorithm falls under the study of compressive sensing, an active area that has recently emerged.
A promiment estimator is The Dantzig Selector (DS) proposed by Tao and Cand\'es, and it is DS which was first applied in channel estimation context.
However, Orthogonal Matching Pursuit (OMP) later emerges as a low-complexity solution, and following-up literature usually used OMP rather than DS as a result.

In this article, we utilize a hybrid structure at both transmitter and receiver end, each of them having both digital and analog percoders and combiners.
We argue in favor of DS, and verify that DS is, as originally perceived, ideal in certain senses.
DS encompasses a more general setting than OMP, such as in its requirement of restricted isometry property (RIP) only, and as in its allowance of noise corruption.
Moreover, we extend their proof to derive a stronger bound in the setting of estimation of millimeter wave channel for almost-sparse channel matrix

We show explicitly that DS may be cast as a second order cone problem (SOCP), and primal-dual technique is ready to be used.
Simulation is done with both DS and OMP, and we discuss their respective merits as our conclusion.


\stoptitle
