
We follow the argument outlined in ``The Dantzig Selector'' and fill the content.
The sparse vector to be recovered of our investigation is \(g \in \MB{C}^{N_h^2}\), and we continue to use \(\hat{g}\) to denote the Dantzig Selector.
To simplify the exposition, we will continue to use the variables \(\V{P}\), \(\V{g}\), \(\V{y}\), and \(\V{z}\) introduced in previous sections, without defining them over and over (by which one would make each statement completely self-containing but considerably more verbose).

We first establish a technical lemma that shows \(\RM{supp}(g)\) has roughly cardinal \(L\).
At the heart of the argument, the lemma provides the basis on which \(g\) is almost-sparse.

\subsection{Beamspace Channel is Almost Sparse}

Consider
%
\Disp{
d =g -\hat{g}
}
%
And let
\begin{itemize}
\item \(\MC{A}\) be the largest \(L\) position of \(g\).
\item \(\MC{B}\) be the largest \(L\) position of \(d\) except on \(\MC{A}\).
\item \(\MC{C} =T(N_h^2) -\MC{B} -\MC{C}\), i.e., the complementary index set.
\end{itemize}
So \(\MC{A}, \MC{B}, \MC{C}\) are mutually exclusive index sets by construction:
%
\Disp{
\MC{A}, \MC{B}, \MC{C} \subset T(N_h^2) \\
\MC{A} \cap \MC{B} =\MC{B} \cap \MC{C} =\MC{A} \cap \MC{C} =\varnothing \\
\MC{A} \cup \MC{B} \cup \MC{C} =T(N_h^2).
}


Recall the definition () in section 1.

\Result
{Definition}
{
Suppose \(\M{A} \in \MB{M}_{\MB{K}} (N_1, N_2)\).
Let \(\MC{T} \subset T (N_2)\).
Denote as \(\M{A}_{\MC{T}}\) the columns of \(\M{A}\) having indices in \(\MC{T}\).
}

\Result
{Definition}
{
The function \(S:\; \MB{V}_{\MB{K}}(N) \mapsto \MB{V}_{\MB{K}}(N)\), \(N \in \MB{N}\), sorts the vector \(x \in \MB{V}_{\MB{K}}(N))\), so that
\Disp{
|S (x)(n_1)| \geq &|S (x)(n_2)|, \\
n_1 >&n_2,
\quad n_1, n_2 \in T (N) \NT
}
}

Note that there are infinitely many sorting functions depending on how to break the tie, but we fix any \(S\) throughout this article.
For definiteness, we may compare the first component first, and the second, and so on.
If still there is a tie for a certain component (for complex numbers), we compare the real part first, then the imaginary part.


\Result
{Definition}
{
Let function \(f:\; T (N) \mapsto \MB{R}_+\), \(N \in \MB{N}\), be given.
For \(x \in \MB{V}_{\MB{K}}(N)\), we say \(x\) is confined in sorted magnitude according to \(f\), if
\Disp{
|S(x) (n)| <&f (n),
\quad n \in T (N).
}
}

\Result
{Lemma}
{
Let \(\varphi (\omega_h)\) be uniformly, independently distributed in \([0,2\pi)\), and array response \(\V{a} (\varphi)\) defined as in ().
Then
%
\Disp{
\| \V{a} (\varphi) \|_1
\leq \F{1}{\pi^2} \log \F{L}{N}.
}
%
}

To find the \(l\)-th path's contribution to the frequency domain, recall the identity for Dirichlet kernel
%
\Disp{
D (\varphi')
:=&\left| \sum_{n=0}^{N-1} \RM{e}^{i n \varphi'} \right| \\ \NT
=&\F{|\sin (N \varphi'/2)|}{|\sin (\varphi' /2)|},
\quad 0 \leq \varphi' \leq \pi
}
%

Recall the Taylor expansion of sin for the first two terms with respect to zero
%
\Disp{
\sin x =x -\F{x^3}{6} + \MC{O}(x^5), \quad x \ll 1
}
%
Actually it can be shown that we have a inequality
%
\Disp{
\left| x -\F{x^3}{6} \right| \leq \sin x, \quad -\pi \leq x \leq \pi
}
%
Applying the above inequality to the denominator and bounding the nominator by 1, we have
%
\Disp{
\left| D (\varphi') \right|
= \F{48}{|\varphi'^2 -24| |\varphi'|}
\quad 0 \leq \varphi' \leq \pi,
}
%
The fraction above can be bounded.

Particularly,
%
\Disp{
\F{1}{N} S \left( \varphi' -\F{2\pi k}{N} \right)
=&\F{1}{N} \left| \sum_{n=0}^{N-1} \RM{e}^{-2\pi i k n/N} \RM{e}^{i n \varphi} \right| \NT \\
\leq& \F{48}{N \pi^3 (\varphi'^3 -24 \varphi' /\pi^2)}
}
%
where \(\varphi' =k/N -\varphi/2\pi\), and the cubic expression is split.
We wish to bound the tail of \(S (\pi \varphi')\), and apply the na\"ive bound
%
\Disp{
&\F{N}{2\pi} \cdot \F{48}{N \pi^3} \int_{L/N}^1 \F{1}{\varphi'^3 -24 \varphi' /\pi^2}
 d\varphi' \NT \\
=&\F{1}{2\pi^2} \cdot \left( 2 \log \F{L}{N} +\log \F{24-\pi^2}{24-\pi^2 L^2 /N^2} \right) \NT \\
\leq& \F{1}{\pi^2} \log \F{L}{N}
}
%
Here, the integral in the first line is immediate.
The second term in the bracket is smaller than 0 and is dropped in the last inequality.
The proof of Lemma () is complete.

The bound is satisfying enough, though refinement is surely possible.
Indeed, \(\log L/N \to 0\) as \(N \to \infty\) and limited by the number of paths \(L\).
Intuitively, this says that when there are few paths, but many antennas, we expect the spatial frequency domain of the virtual channel to be sparse too.

\subsection{Design of RIP Precoders and Combiners}

\subsection{DS for Complex Vectors}

\Result
{Definition}
{
For \(\MC{T}, \MC{T}' \subset \{1, \dotsc, N_p\}\), define the \(s,s'\)-restricted orthogonality constant \(\tau_{s,s'} >0\) to be the smallest such that
%
\Disp{
| \IP{ P_{\MC{T}} h, P_{\MC{T}'} h' } |
\leq \tau_{s, s'} \cdot \| h \|_2 \|h'\|_2
}
}

From ``Decoding from Linear Programming'',
\Result
{Lemma}
{
%
\Disp{
\leq 
\tau_{s, s'}
\leq \delta_s
}
}

The rest follows ``The Dantzig Selector'' very closely.
\Result
{Proposition}
{
Let \(g\) and \(d\) be defined as above.
Then
%
\Disp{
\|d_{BC}\|_1
\leq \|d_{\MC{A}}\|_1 +\|g_{BC}\|_1
}
}

To show this, observe
%
\Disp{
&\|g\|_1 -\|d_{\MC{A}}\|_1 +\|d_{BC}\|_1 -\|g_{BC}\|_1 \\
\leq& \|g +d\|_1 \NT \\
\leq& \|g\|_1
}
%
In the first line, we recall \(g =g_{\MC{A}}\).

\Result
{Lemma}
{
Let \(d\) and \(P\) be defined as above, and \(P\) satisfies RIP as in ().
%
\Disp{
|\IP{ d, P(:, n_h) }|
\leq& 4 \sqrt{\log N_h},
\quad n_h =1, \dotsc, N_h^2.
}
%
}

\Result
{Lemma}
{
Let \(d\) and \(P\) be defined as above, and \(P\) satisfies RIP as in ().
%
\Disp{
\|P^\H P d\|_\infty
\leq& 4 \sqrt{\log N_h},
\quad n_h =1, \dotsc, N_h^2.
}
%
}

To show this, start with
%
\Disp{
&\IP{ z -(y -P g), P (:,n_y) } \NT \\
=&\IP{ P \hat{g} -P g, P (:,n_y) } \NT \\
=&\IP{ P g, P (:,n_y) }
}
%
Result follows from triangle inequality, and Lemma ().

From ``Dantzig Selector'' (Cand\`es and Tao 2007), Lemma 1, first equation, we have the result as below.
The original result is for real vector spaces, but we have checked that the proof is completely valid in complex vector spaces.
Of course the interpretation of modulus and the inner product changes accordingly.

\Result
{Lemma}
{
Let \(d\) and \(P\) be defined as above, and \(P\) satisfies RIP as in () with constant \(\delta\).
%
\Disp{
\|d_{\MC{A}}\|_2
\leq \F{1}{1-\delta} \|P^\H_{\MC{A}\MC{B}} P d\|_2 +\F{\delta}{(1-\delta) \sqrt{L}} \|d_{BC}\|_1
}
%
}

From ``Dantzig Selector'' (Cand\`es and Tao 2007), Lemma 1, second equation, we have the result as below.
Again, their proof works with complex vector spaces in place of real ones too.

\Result
{Lemma}
{
Let \(d\) be defined as above.
Then
%
\Disp{
\|d\|_2^2
\leq \|d_{\MC{A}\MC{B}}\|_2^2 +\F{1}{L} \|d_{\MC{C}}\|_1^2
}
%
}

\subsection{The Main Bound}

We can now formulate and prove the main bound of ours.

{ \color{red} (To be done) }
