\Frame {The Need of Bounding the Expected Error}
{
\I Dantzig Selector requires the measurement matrix to observe restricted isometry property (more below), but it is not clear it is true for \m {\M {P}}, for what design of \m {\M {F}_B, \M {F}_R, \M {W}_B, \M {W}_R}.

\I Cand\`es and Tao established the case for real matrices, but it is not clear whether complex case applies.

\I Cand\`es and Tao showed the Big-O error bound in terms of (in our notation) \m {N_H}, but it is worth investigation whether, for practical values, the algorithm converges quickly enough.
}

\Frame {Restricted Isometry Property}
{
\I \m {\V {x}} is called \m {s}-sparse, if only \m {s} of its components are nonzero.

\I We say that \m {\M{\Phi}} satisfies the restricted isometry property (RIP) of sparsity \m {s} with respect to \m {0 \leq \d_s \leq 1}, if, for all \m {s}-sparse \m {\V {x}},
\Disp {
\NC \RB {1-\d_s} \VNm {\V {x}}^2
\leq \NC \VNm {\M{\Phi} \V {x}} _2^2
\leq \RB {1+\d_s} \VNm {\V {x}} _2^2
}

\I Interpretation: \m {\M{\Phi}} is \quotation {almost unitary} up to \quotation {relative error} \m {\d_s}.
}
% % % % % % % % % % % % % % % % % % % % % % % % % %
\Frame {Proof Strategy}
{
\I Show that \m {\V{g}} is almost sparse

\I Find \m {\MB {E} \SB {\VNm {\M{P} \V{u}} _2 ^2}} and \m {\MB {E} \SB {\VNm {\M{P} \V{u}} _2 ^4}}

\I Bound \m {\MB {E} \SB {\VNm {\M{P} \V{u}} _2 ^k}} for \m {k=6,8,10,\dots}.

\I Bound the probability that \m {\Nm {\VNm {\M{P} \V{u}} _2 ^2 -\VNm {\V{u}} _2 ^2} \geq \e \VNm {\V{u}} _2 ^2} by Chernoff Bound.

\I Established that \m {\M{P}} satisfies RIP for such probability by invoking existing results

\I Substitute \m {\VNm {\V {g} _{\MC {K}}} _1} into the original Dantzig Selector Proof
}
% % % % % % % % % % % % % % % % % % % % % % % % % %
\Frame {\m {\V{g}} is Almost Sparse}
{
\I Let \m {s} be a small integer.
Call the vector of largest-magnitude \m {s^2 L} positions of \m {\V{g}} to be \m {\V{g}_A}, and remaining positions of \m {\V{g}} to be \m {\V{g}_K}, so that \m {\V{g}_A +\V{g}_K =\V{g}}

\I We can show that 
\Disp {
\NC \V{g}_K
\leq R
:= \NC \VNm {\MC {C} \SB {\M {K}^\Adj \V {a} \SB {\f}, s}} _2
\leq \m {\F{1}{3 \R{N_H}}} \NR
}
}
% % % % % % % % % % % % % % % % % % % % % % % % % %
\Frame {Design of \m {\M{F}_B}}
{
\I Set each entry of \m {\M{F}_B} to be i.i.d.\ Gaussian r.v.\ with mean 0, standard deviation \m {1/2}, multiplied by \m {\l_B >0}

\I The magnitude \m {\M{F}_B \DB {n_R, n_Y}} follows Rayleigh distribution, having
\Disp {
\NC M_{B,2}
:=\NC \MB{E} \SB {\Nm {\M{F}_B \DB {n_R, n_Y}}^2}
=\l_R^2 \NR
\NC M_{B,4}
:=\NC \MB{E} \SB {\Nm {\M{F}_B \DB {n_R, n_Y}}^4}
=2 \l_R^4 \NR
}

\I We have \m {\MB{E} \SB {\M{F}_B \DB {n_R, n_Y}^2} =0}
}
% % % % % % % % % % % % % % % % % % % % % % % % % %
\Frame {Design of \m {\M{F}_R}}
{
\I Let \m {\M{F}_R \DB {n_H, n_R}} be uniformly distributed on the unit circle on the complex plane, giving
\Disp{
\NC M_{R,2}
:=\NC \MB{E} \SB {\Nm {\M{F}_R \DB {n_H, n_R}}^2}
=\l_R^2 \NR
\NC M_{R,4}
:=\NC \MB{E} \SB {\Nm {\M{F}_R \DB {n_H, n_R}}^4}
=\l_R^4 \NR
}

\I We have \m {\MB{E} \SB {\M{F}_R \DB {n_H, n_R}^2} =0}
}
% % % % % % % % % % % % % % % % % % % % % % % % % %
\Frame {Indicator Function}
{
\I To make expressions more compact, introduce the indication function \m {\i}, which \m {=1} only when the manner that arguments repeat agrees exactly with the subscripts

\I For example \m {\i_{2} \SB {7,7} =1}, and \m {\i_{2,2} \SB {0,3,3,0} =1}, but \m {\i_{2} \SB {5, 6} =0}

\I That is, for some injective \m {\s:\; \CB {0, \ldots, M-1} \mapsto \CB {0, \ldots, N-1}},
\Disp {
\NC \NC \i_{a_0, \ldots, a_{N-1}} \SB {x_0, \ldots, x_{M-1}} \NR
\NC =\NC \startcases
\NC 1, \MC \Q
x_0 \cdots x_{N-1}
=x_{\s \SB {0}} ^{a_0} \cdots x_{\s \SB {M-1}} ^{a_{M-1}} \NR
\NC 0, \NC \Q \Rm {otherwise} \NR
\stopcases \NR
}

}
% % % % % % % % % % % % % % % % % % % % % % % % % %
\Frame {Suffice to Ignore \m {\M{K}}}
{
\I Recall \m {\M {P} := \RB {\M {F}_B^\Tr \M {F}_R^\Tr \M{K}^\ast} \otimes \RB {\M {W}_B \M {W}_R \M {K}}}

\I Define \m {\M{Q} :=\RB {\M {F}_R^\Tr \M {F}_B^\Tr} \otimes \RB {\M {W}_B \M {W}_R}}, then
\Disp {
\NC \V{u} ^\Adj \M{P} ^\Adj \M{P} \V{u}
= \NC \V{u} ^\Adj \M{Q} ^\Adj \M{Q} \V{u} \NR
}

\I Thus
\Disp {
\NC \VNm {\M{P} \V{u}} _2
= \NC \VNm {\M{Q} \V{u}} _2 \NR
}
}
% % % % % % % % % % % % % % % % % % % % % % % % % %
\Frame {Finding \m {\MB {E} \VNm {\M{P} \V{u}} _2 ^2}}
{
\Disp{
\NC \NC \MB{E} \SB {\VNm {\M{Q} \V{u}} _2 ^2} \NR
\NC = \NC \MB{E} \SB {
  \sum_{n_y=0}^{N_y-1}
  \Nm {\sum_{n_h=0}^{N_h-1} \M{Q} \DB {n_y, n_h} ^\ast \V{u} \DB{n_h}} ^2
} \NR
\NC = \NC
2 \sum_{n_y=0}^{N_y-1}
\sum_{\Stack { n_h, n_h'=0 \NR n_h <n_h' }}^{N_h-1}
\MF {Re} \SB{
   \MB{E} \SB {\M{Q} \DB {n_y, n_h}^\ast}
   \MB{E} \SB {\M{Q} \DB {n_y, n_h'}}
}
\V{u} \DB{n_h} \V{u} \DB{n_h'} ^\ast
\NR
\NC \NC \FourQ +\sum_{n_y=0}^{N_y-1}
   \sum_{n_h=0}^{N_h-1} \MB{E} \SB {\Nm {\M{Q} \DB {n_y, n_h}}^2} \Nm {\V{u} \DB{n_h}}^2
\NR
\NC = \NC N_y \D N_R^2 \l_B^4 \l_R^4  \D \VNm {\V{u}} _2 ^2 \NR
}
}
% % % % % % % % % % % % % % % % % % % % % % % % % %
\Frame {Confirming the RIP of \m {\M{P}}}
{
\I Finding the variance
\Disp{
\NC \Ss{Var} \SB {\VNm {\M{P} \V{u}} _2 ^2}
\leq \NC \RB {4 N_H^2 N_Y^{-3} -1} \VNm {\V{u}} _2 ^4 \NR
}

\I Using the Chebyshev Bound
\Disp{
\NC \MB{P}
\SB {
  \Nm {\VNm {\M{P} \V{u}} _2 ^2 -\VNm {\V{u}} _2 ^2}
  \geq \e \VNm {\V{u}} _2 ^2
}
\leq \NC \F {1} {\e^2} \RB {4 N_H^2 N_Y^{-3} -1} \NR
}
}


% % % % % % % % % % % % % % % % % % % % % % % % % %
\Frame {Expected Error of DS}
{
\I Let \m {\V {y}}, \m {\M {P}}, \m {\V {g}}, \m {\hat {\V {g}}}, \m {\V {d}} be defined as above.
Then, with \m {S =s^2 L},
\Disp {
\NC \VNm {\V {d}} _2
\leq \NC 32 s^2 L \log N_H \NR
}

\I Moreover, under the design condition
\Disp {
\NC \RB {2 N_H}^2
\approx \NC N_Y^3 \NR
}

\I The bound holds for probability \m {p}, with
\Disp {
\NC 1 -p
\leq \NC 16 \D {12}^S \d_{S}^{-S-2} \RB {N_H^2 N_Y^{-3} -\F{1}{4}} \NR
}

\I \m {\Rightarrow} Is \m {16 \D {12}^S \d_{S}^{-S-2}} unreasonably large?

\I \m {\Rightarrow} Can \m {N_H^2 N_Y^{-3} -1/4} be made small?
}
% % % % % % % % % % % % % % % % % % % % % % % % % %
\Frame {Technical Lemmata}
{
{\tfx
\I Define \m {\V {d} := \Hat {\V {g}} -\V {g}}

\I From optimality of \m {\Hat {\V {g}}}: \m {
\NC \VNm {\V {d} _{\MC {K}}} _1
\leq \NC \VNm {\V {d} _{\MC {A}}} _1 +2\VNm {\V {g} _{\MC {K}}} _1 \NR
}

\I From central limit theorem: \m {
\NC \MB {E} \SB {\Nm {\IP {\V {z}, \M {P} \DB {:, n_h}}}}
\leq \NC  2 \R {\log N_h} \NR
}

\I From above: \m {
\NC \VNm {\M {P}^\Adj \M {P} \V {d}} _\infty
\leq \NC  4 \R {\log N_h} \NR
}

\I From DS paper: \m {
\NC \VNm {\V {d} _{\MC {AB}}} _2
\leq \NC \F {1}{1-\d_{2S}} \VNm {P _{\MC {A} \MC {B}}^\Tr P d} _2 +\F {\d_{3S}}{\RB {1-\d_{2S}} \R {S}} \VNm {d_{\MC {K}}} _1 \NR
}

\I From DS paper: \m {
\NC \VNm {\V {d}} _2^2
\leq \NC \VNm {\V {d} _{\MC {A} \MC {B}}} _2^2 +\F {1}{S} \VNm {\V {d} _{\MC {K}}} _1^2 \NR
}
}
}
% % % % % % % % % % % % % % % % % % % % % % % % % %
\Frame {Proof (1 of 2)}
{
\I By \m {\ell_p}-norm ineq., by Lemma for \m {\VNm {\M {P}^\Tr \M {P} \V {d}} _\infty},
\Disp {
\NC \VNm {\M {P} _{\MC {A} \MC {B}}^\Tr \M {P} \V {d}} _2
\leq \NC \R {S} \VNm {\M {P}^\Tr \M {P} \V {d}} _\infty
\leq 4 \R {S \log N_h} \NR
}

\I By \m {\ell_p}-norm ineq., by Lemma for \m {\VNm {\V {d} _{\MC {AB}}} _2}, by above,
\Disp {
\NC \VNm {\V {d}_{\MC {A}}} _1
\leq \NC \R {S} \VNm {\V {d}_{\MC {A}}} _2 \NR
\NC \leq \NC \d_{3S} \RB {1+\d_{2S}} \VNm {\V {d} _{\MC {K}}} _1
+4 \RB {1+\d_{2S}} S \R {\log N_h} \NR
}

\I By Lemma for \m {\VNm {\V {d} _{\MC {K}}} _1}, by above, by dropping terms,
\Disp {
\NC \VNm {\V {d} _{\MC {K}}} _1
\leq \NC 2 \RB {1+\d_{3S}} \VNm {\V {g} _{\MC {K}}} _1
+4 \RB {1+2\d_{3S}} S \R {\log N_h} \NR
}
}
% % % % % % % % % % % % % % % % % % % % % % % % % %
\Frame {Proof (2 of 2)}
{
\Disp {
\NC \VNm {\V {d}} _2^2
\leq \NC 16 S \log N_h
+8 \d_{3S} \R {\log N_h} 
+\F {\d_{3S}^2} {S} \VNm {\V {d} _{\MC {K}}} _1 \NR
\NC \leq \NC 32 s^2 L \log N_H
+\F {8 \R {2\pi}} {3} \d_{3S} L \R {\F {\log N_H} {N_H}}
+\F {\pi} {9} \F {L} {s^2 N_H} \NR
\NC \leq \NC 32 s^2 L \log N_H \NR
}

1. Plug in the bound for \m {\VNm {\V {d} _{\MC {K}}} _1}

2. Plug in \m {S =s^2 L} and \m {N_h =N_H^2}
}


