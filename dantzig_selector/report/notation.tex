\starttitle [title={Notation}]

To avoid confusion, round brackets \m {\RB {\cdot}} are used purely for precedence of , and \m {\SB {\cdot}} exclusively for function arguments, for better clarity.

\m {\Nm {a}} denotes the magnitude for real or complex numbers \m {a}.
The real and imaginary part of \m {a} will be denoted respectively as \m {\MF {Re} \SB {a}} and \m {\MF {Im} \SB {a}}, respectively, thus \m {a =\MF {Re} \SB {a} + \Ss{i} \MF {Im} \SB {a}}

We restrict our discussion to \m {\MB {R}^N} or \m {\MB {C}^N} seen as the Hilbert space , equipped with standard inner product.
Lower case Latin alphabets with single underline, like \m {\V {a}}, denote vectors, and upper case Latin alphabets with double underlines, like \m {\M {A}}, denote matrices.
Particularly, define \m {\V {1} =\IP {1, \dots, 1}}.
Denote \m {\V {a} \DB {n}} to be the \m {n}-th component of \m {\V {a}}, and similarly \m {\M {A} \DB {m,n}} to be the \m {m,n}-th entry of \m {\M {A}}.

In our context, the complex numbers has natural interpretation as the phasor of electronic waves.
Of course, scalar operation on \m {\MB {R}^N} is over field \m {\MB {R}}, and that on \m {\MB {C}^N} is over field \m {\MB {C}}.
We use \m {\MB {K}} to refer to either \m {\MB {R}} or \m {\MB {C}}.
Not to avoid confusion, we call the Hilbert space \m {\MB {K}^N} over \m {\MB {K}} to be \m {\MB {V}_{\MB {K}} \SB {N}}.
We also call the collection of \m {M} by \m {N} matrices to be \m {\MB {M}_{\MB {K}} \SB {M,N}}, and its role as linear transformation is understood.
Inner product of \m {\V {a}} and \m {\V {b}} is denoted as \m {\IP { \V {a}, \V {b} }}.
Slightly abusing notation, we denote as \m {\V {a} \DB {n}} the \m {n}-th component of \m {\V {a}}, and denote as \m {\V {A} \DB {m,n}} the \m {m,n}-entry of \m {\V {A}}
To better correspond to the pseudo code, the numbering of both vector and matrix entries are zero-based, that is, numbered from \m {0,1,2,\dots}.
Especially when manipulating the indices, but also in the main proof of DS bound, we will encounter the modulo as operation, namely the remainder of \m {m} divided by \m {n}, with \m {m, n \geq 0}, and we use the capital \m {m\; \Rm{Mod}\; n} to denote that number.
And we denote \m {\V {u}_n} to be the \m {n}-th unit vector in the Cartesian coordinates, whose numbering is also zero-based.
\m {\VNm {\V {a}} _p} denotes the \m {\ell_p}-norm of vector \m {\V {a}}.
Recall that \m {\ell_\infty}-norm is the maximum-magnitude norm.

Denote as \m {\MB {P} \SB {X}} the probability, \m {\MB {E} \SB {X}} the expectation, and \m {\Ss {Var} \SB {X}} the variance.
We may spell out the random variables' dependence of event space index when confusion might arise, as in \m {X \SB {\o}}.

\stopchapter
