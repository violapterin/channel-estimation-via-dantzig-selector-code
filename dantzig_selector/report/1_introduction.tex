\startchapter [title={Introduction}]

This section is organized as follows.
Before we start, we introduce our notation in the very beginning, since the literature survery quickly goes into technical details.
And we first provide the motivation, and then overview the existent literature on channel estimation.
Particularly, we will summerize present results on DS and OMP, and argue that DS is a viable solution that has been ignored, and then claim our contribution in this article that will be shown in the later sections.

\startsection [title={Notation}]

Before we start, let us remind that, in this article, to avoid confusion all results other than the major theorems to be proved are called lemmata, so the classification may be different from its source.

First of all, round brackets \m {\RB {\cdot}} are used purely for precedence of , and \m {\SB {\cdot}} exclusively for function arguments, for better clarity.

\m {\Nm {a}} denotes the magnitude for real or complex numbers \m {a}.
The real and imaginary part of \m {a} will be denoted respectively as \m {\Mf {Re} \SB {a}} and \m {\Mf {Im} \SB {a}}, respectively, thus \m {a =\Mf {Re} \SB {a} + i \Mf {Im} \SB {a}}

We restrict our discussion to \m {\Mb {R}^N} or \m {\Mb {C}^N} seen as the Hilbert space , equipped with standard inner product.
Lower case Latin alphabets with single underline, like \m {\V {a}}, denote vectors, and upper case Latin alphabets with double underlines, like \m {\M {A}}, denote matrices.
Particularly, define \m {\V {1} =\IP {1, \dots, 1}}.
Denote \m {\V {a} \DB {n}} to be the \m {n}-th component of \m {\V {a}}, and similarly \m {\M {A} \DB {m,n}} to be the \m {(m,n)}-th entry of \m {\M {A}}.

In our context, the complex numbers has natural interpretation as the phasor of electronic waves.
Of course, scalar operation on \m {\Mb {R}^N} is over field \m {\Mb {R}}, and that on \m {\Mb {C}^N} is over field \m {\Mb {C}}.
We use \m {\Mb {K}} to refer to either \m {\Mb {R}} or \m {\Mb {C}}.
Not to avoid confusion, we call the Hilbert space \m {\Mb {K}^N} over \m {\Mb {K}} to be \m {\Mb {V}_{\Mb {K}} \SB {N}}.
We also call the collection of \m {M} by \m {N} matrices to be \m {\Mb {M}_{\Mb {K}} \SB {M,N}}, and its role as linear transformation is understood.
Inner product of \m {\V {a}} and \m {\V {b}} is denoted as \m {\IP { \V {a}, \V {b} }}.
Slightly abusing notation, we denote as \m {\V {a} \DB {n}} the \m {n}-th component of \m {\V {a}}, and denote as \m {\V {A} \DB {m,n}} the \m {\DB {m,n}}-entry of \m {\V {A}}
To better correspond to the pseudo code, the numbering of both vector and matrix entries are zero-based, that is, numbered from \m {0,1,2,\dots}.
Especially when manipulating the indices, but also in the main proof of DS bound, we will encounter the modulo as operation, namely the remainder of \m {m} divided by \m {n}, with \m {m, n \geq 0}, and we use the capital \m {m\; \rm{Mod}\; n} to denote that number.
And we denote \m {\V {u}_n} to be the \m {n}-th unit vector in the Cartesian coordinates, whose numbering is also zero-based.
\m {\VNm {\V {a}} _p} denotes the \m {\ell_p}-norm of vector \m {\V {a}}.
Recall that \m {\ell_\infty}-norm is the maximum-magnitude norm.

Denote as \m {\Mb {P} \SB {X}} the probability, and \m {\Mb {E} \SB {X}} the expectation.
Since our discussion involving the entire communication system will get very complicated, we may spell out the dependence of event space index on random variables.

\stopsection
\startsection [title={Background}]

Multiple-input multiple-output (MIMO) communication system has been proposed to be the next-generation specification.
With a large number of antennae of both transmitter and receiver sides (hereafter massive MIMO) is therefore expected to improve spectral efficiency.
But MIMO presents new obstacles as well.
The hardware overhead due to large number of antennas increases complexity and power consumption, and new precoders are being invented to take account of this.
It is important to devise a feasible scheme, as the systems themselves are growing more complicated.
Meanwhile, millimeter waves, having smaller wavelength, entails higher frequency bands, and thus wider available bandwith to be available.
In addition, the smaller closer spaced antennae makes it possible to increase the number antennae.
Consequently, this article considers MIMO in the millimeter r\`egime.

The figure of merit of a MIMO system is usually considered to be the channel capacity.
In vague terms, the capacity \m {C} of a MIMO channel \m {\M {H}} at perfect channel state information is well-known to be roughly in the form \m {\log \det \RB {I +\Mr {SNR}}}, but that knowledge is hardly always available.
What makes things more difficult is the case that \m {\M {H}} is determined from not only the noise itself, but also inherent parameter --- such as, in our case, as shall be seen, the angle of departure and arrival.
It is pointed out that the capacity written in explicit dependence of such parameters is a difficult and long-standing problem (Goldsmith et.\ al.\ 2003).
Nevertheless, the perfect-channel-state-information expression of channel capacity is often used regardless of these issues.
Indeed, analysis of precoding algorithm, for example, usually makes use of the sum rate \m {\log \det \RB {I +\Mr {SNR}}}.

Therefore the real-time estimation, \m {\hat {\M {H}}}, of \m {\M {H}} is of paramount importance.
Needless to say, when the \m {\hat {\M {H}}} is imprecise, resulting analysis is also undermined.
With more antennae present, conventional training-based algorithms also increase in complexity and storage.
In real applications, channel may also be fast varying with respect to time, and a high complexity algorithm is surely less than being ideal.
A design of new algorithms that addressed these issues is thus necessary.

\stopsection
\startsection [title={The Dantzig Selector}]

It is a recent developement that the estimation of channels is facilitated by advances of compressed sensing.
A series of paper by Cand\`es and Tao (2006) marks its advent.

To motivate their work, note that it is a common situation in statsitical applications that the number of model parameters \m {N_p \in \Mb {N}} is much larger than the number of measurements \m {N_m \in \Mb {N}}, namely,

\Disp {
\NC N_p \gg N_m \NR
}

In such case of insufficient measurements, possibly even with corruption of noise, do we have the knowledge of all \m {N_p} parameters?
Of course, more assumption must be made to make the question meaningful.
Any two vectors of parameters differing an element in the null space of the sensing matrix will produce the same measurement.

Cand\`es and Tao showed frst that that in the noiseless case (``Decoding by Linear Programming'', 2006), it is possible to recover the sparse signal, under a \m {\ell_1} minimization program.
The work reveals the phenomenon that few observations of the signal in question may be sufficient for us to resonstruct the signal when it is sparse in a certain sense.

Later they established a stronger result that (``The Dantzig Selector'', 2007), with another \m {\ell_1} minimization program they named The Dantzig Selector (hencefore DS), they can recover the noisy case too.
Such recovery of signals is possible, if we make a few carefully constructed, and seemingly random measurements.
Ever since, it becomes feasible that a camera equipped with few sensors may obtain high quality images, greatly reducing the subsequent cost.
In addition, they derived a probability bound of mean suqare error, showing the probability of successful recovery is overwhelming.

From the realistic perspective, the formulation as an \m {\ell_1} minimization problem, which is convex, made techniques from convex optimization usable.
Indeed, relevent code has been put on the web for the reader to access and verify (Cand\`es \& Romberg 2005).

In this article we focuss on DS, and we introduce several notions before we describe their work.
To facilitate reader's understanding, in this section we will consistently use \m {N_p} and \m {N_m} to denote matrix dimensions.
Following Cand\`es and Tao, we borrow the term ``support'' from the context of analysis, where it denotes the set on which a function is nonzero.

For brevity, we will use \m {T \RB {N_m} =\{1, \dots, N_m\}} to denote the support of \m {\Mb {V}_{\Mb {K}} \SB {N_m}}.
Also, suppose \m {\M {A} \in \Mb {M}_{\Mb {K}} \SB {N_m, N_p}}.
Let \m {\Mc {T} \subset T \RB {N_p}}.
Denote as \m {\M {A}_{\Mc {T}}} the columns of \m {\M {A}} having indices in \m {\Mc {T}}.

\Result
{Definition}
{
\m {x \in \Mb {V}_{\Mb {K}} \SB {N_m}} is called \m {s}-sparse with support \m {\Mc {A}}, \m {s <N_m}, if there is \m {\Mc {A} \subset T \RB {N_m}} such that

\Disp { 
\NC x_{T\SB {N} -\Mc {A}}
=\NC \V {0} \NR
}
with
\Disp { 
\NC \# \Mc {A} \leq s. \NR
}
That is, only at most \m {s} components of \m {x} is nonzero.
}

\Result
{Definition}
{
Consider \m {P \in \Mb {M}_{\Mb {K}} \SB {N_m, N_p}}, with unity-\m {\ell_2}-norm columns.
For \m {s =1, \dots N_p}, we say that \m {P} satisfies the restricted isometry property (RIP) of sparsity \m {s} with respect to \m {0 \leq \d_s \leq 1}, if, for all \m {s}-sparse \m {\V {x} \in \Mb {V}_{\Mb {K}} \SB {N_p}}, for all \m {\Mc {T} \subset T\SB {N_m}} with \m {\#\{\Mc {T} \} \leq s},

\Disp {
\NC \NC \RB {1-\d_s} \VNm {V {x}}^2 \NR
\NC \leq \NC \VNm {P_{\Mc {T}} \V {x}} _2^2 \NR
\NC \leq \NC \RB {1+\d_s} \VNm {\V {x}} _2^2 \NR[+]
}
}

RIP is essentially saying that \m {P} is ``almost unitary'' up to ``relative error'' \m {\d_s}.

Now, consider a linear transformation with noise corruption,

\Disp {
\NC \V {y} =\M {P} \V {h} + \V {z} \NR
}

where (w.l.o.g.) \m {z} is an i.i.d.\ normalized Gaussian vector.
How can we hope to estimate \m {h} when, in addition to insufficient observations, there are too few observations?

\Result
{Algorithm}
{
\startitemize[n]
\item Input \m {\M {P} \in \Mb {V}_\Mb {R} \SB {N_m, N_p}}, \m {\V {y} \in \Mb {V}_\Mb {R} \SB {N_m}}.
\item Find \m {\V {h} \in \Mb {V}_\Mb {R} \SB {N_m}} by
\Disp {
\NC \V {h}
= \NC \startcases
\NC \Min {\V {h}'} \quad \MC \VNm {\V {h}'} _1 \NR
\NC \Mr {subject} \; \Mr {to} \quad \MC \VNm {\M {P}^\Adj \RB {\M {P} \V {h} -\V {y}}} _\infty \leq \g \NR
\stopcases \NR
}
\item Output \m {\V {h}}.
}

They was able to show that, with Algorithm (), the mean square error is bounded with overwhelming probability.
And it can be shown that this \m {\ell_1}-minimization problem with \m {\ell_\infty}-constraint may be recast as a linear program (LP), lending convex programming technique applicable.
Note that they confined their discussion for real vectors.

\stopsection
\startsection [title={Sufficient Conditions of RIP}]

The constant \m {\d} in the proof of DS is significant.
Indeed, we may observe that in theorem (), the bound is not tightest when \m {\d=0}, i.e., that \m {P} is fully unitary, rendering the RIP an important role.
Here, it is not clear at first what matrices serves as the RIP condition.

But later, Baraniuk (2008) have found there is a particularly convenient sufficient condition to verify RIP.

\Result
{Theorem}
{
Let each entry of random matrix \m {P \in \Mb {M}_\Mb {K} \SB {N_m, N_p}} be defined as i.i.d.\ Bernoulli with value \m {\pm 1/\R {N_m}} (that is, Rademacher distribution).
Then, with \m {0 <\e <1},

\Disp {
\NC \NC \Mb {P} \SB { \Nm {\VNm {P h} _2^2 -\VNm {h}_2^2} \geq \e \VNm {h} _2^2} \NR
\NC \leq \NC 2 \exp \SB {-n \RB {\F {\e ^2} {4} -\F {\e ^3} {6}} } \NR
}
}

\color[red] {(To be done)}

\stopsection
\startsection [title={Compressed Channel Sensing}]

We have summarised results in compressed sensing, and we now relate the matter to the communication scenario.
Recall that the overhead of channel estimation has been a concern in the millimeter wave MIMO setting.
Meanwhile, physical evidences suggest that millimeter wave channels can be said to be sparse in the frequency domain in a certain senses.
Some scholars thus has applied compressive sensing techniques to the problem.

One of the first papers, Bajwa et.\ al.\ (2010) argues the \m {\ell_0}-norm of the channel matrix may be bounded by a constant, and in such settings the Dantzig Selector may be applied.
What they explored was the estimation of single-antenna channel response with respect to time.
Another paper by the same group of scholars (Bajwa et.\ al.\ 2008) shows that \m {X} is RIP for overwhelming probability, providing the ground for the former paper.

To explain the idea, consider a linear time-invariant channel that \m {y\SB {t} =\RB {x \star h} \SB {t} + z\SB {t}}, where \m {h\SB {t}} is the channel's impulse response, \m {y\SB {t}} is the output, and, for each time index, \m {x\SB {t}} is i.i.d.\ Rademacher, and \m {z\SB {t}} is i.i.d.\ Gaussian.
Simply assuming \m {h} has sparse support, and let the convolution with \m {x} be expressed by a matrix \m {X}, \m {X} is a Toeplitz matrix, so that \m {y =X h +z}.
Then Bajwa et.\ al.\ was able to show that, \m {X} is RIP of overwhelmingly probability, and DS is ready to be applied, with some nice bounds on its performance drawing from Cand\`es and Tao's theorems.

\stopsection
\startsection [title={Orthogonal Matching Pursuit}]

Around that time, Tropp and Gilbert (2007b) points out if we apply a greedy algorithm called Orthogonal Matching Pursuit (OMP), that an i.i.d.\ random matrix being generated the similar way as Baraniuk (2008) suggested, may perform sufficiently well, even recover the original signal in an overwhemingly probability too.

Same as before, consider \m {\V {y} \in \Mb {V}_\Mb {K} \SB {N_p}}, \m {\V {h} \in \Mb {V}_\Mb {K} \SB {N_s}}, where \m {\V {h}} is \m {s}-sparse, with \m {N_p \gg N_s}.
If
\Disp {
\NC \V {y} =\M {P} \V {h}, \NR
}
that is, the noiseless, real-number case.
We pick up the columns of \m {\M {P}} greedily, hoping to correspond to the support of \m {\V {h}}, thus recovering the original signal.

\Result
{Algorithm}
{
\startitemize[n]
\item To be done.
\stopitemize
}

From Tropp and Gilbert (2007a),

\Result
{Theorem}
{
Following the setting described above, suppose, for \m {0 <\d <1},

\Disp {
\NC M \geq 16s \log \RB {N/\d} \NR
}

Then OMP (Algorithm \_\_) recovers \m {x} completely, with overwhelming probability \m {1-\d}.
}

It is interesting to note that, ever since, scholars has applied OMP in favor of DS, due the simplicity of former's implementation, making OMP perhaps the most widely used channel estimation technique.

\stopsection
\startsection [title={OMP used in Channel Sensing}]

For concreteness, let \m {\M {F}} be the precoder and \m {\M {W}} the combiner, and \m {\M {H}} the channel, and consider the noiseless case.
Let signal \m {\V {x}} be sent, and \m {\V {y}} be receiverd, that is

\Disp {
\NC \V {y} =\M {W} \M {H} \M {F} \V {x} \NR
}

% % % % % % % % % % % % % % % % % % % % 

%I want to show the Kronecker Product identity listed on [Wikipedia](https://en.wikipedia.org/wiki/Kronecker_product):
%$$\begin{align}
%\mathrm{vec}(AXB)
%=(B^T \otimes A) \mathrm{vec}(X) \\ \tag{1}
%\end{align}$$
%Wikipedia does not cite references for the proof for (1), and I want to verify.
%Before the calculation, agree with some convention.
%Let us denote $a\; \mathrm{mod}\; b$ for the remainder of $a$ modulo $b$ (so that $9\; \mathrm{mod}\; 4 =1$).
%This is unlike the usual practice, but otherwise the expression gets too verbose.
%Moreover let us denote the $(i,j)$-th entry of $A$ to be $A_{i,j}$, and $i$-th entry of $v$ to be $v_{i}$.
%Also say that $A \in \mathbb{R}^{M \cdot N}$ if it is a $M$ by $N$ matrix, by slightly abusing the meaning of $\mathbb{R}$.
%If we stick to only addition and multiplication, once the proof works, it works for any field.
%
%Firstly, restate the formal definition of Kronecker product and vectorization, for later use.
%For $X' \in \mathbb{R}^{N_2 \cdot N_3}$
%$$\begin{align}
%(\mathrm{vec}(X'))_{m_1}
%=X'_{m_1\; \mathrm{mod}\; N_1, \lceil m_1/N_1 \rceil} \\ \tag{2}
%\end{align}$$
%
%For $A' \in \mathbb{R}^{N_1 \cdot N_2}, B' \in \mathbb{R}^{N_3 \cdot N_4}$,
%$$\begin{align}
%(A' \otimes B')_{m_1, m_2}
%=A'_{\lceil m_1/N_3 \rceil, \lceil m_2/N_4 \rceil}
%B'_{m_1\; \mathrm{mod}\; N_3, m_2\; \mathrm{mod}\; N_4} \\ \tag{3}
%\end{align}$$ 
%
%Now, let $A \in \mathbb{R}^{N_1 \cdot N_2},\; X \in \mathbb{R}^{N_2 \cdot N_3},\; B \in \mathbb{R}^{N_3 \cdot N_4},$,
%We try to evaluate lhs and rhs of (1).
%
%$$\begin{align}
%&(\mathrm{vec}(AXB))_{m_1} \\
%=&(AXB)_{m_1\; \mathrm{mod}\; N_1, \lceil m_1/N_1 \rceil} \\
%=&\sum_{n_2=1}^{N_2} \sum_{n_3=1}^{N_3}
%A_{m_1\; \mathrm{mod}\; N_1, n_2} X_{n_2, n_3} B_{n_3, \lceil m_1/N_1 \rceil} \\ \tag{4}
%\end{align}$$
%
%$$\begin{align}
%&((B^T \otimes A) \mathrm{vec}(X) )_{m_1} \\
%=&\sum_{m_2=1}^{N_2 N_3} (B^T \otimes A)_{m_1, m_2} \mathrm{vec}(X)_{m_2} \\
%=&\sum_{m_2=1}^{N_2 N_3} (B^T)_{\lceil m_1/N_1 \rceil, \lceil m_2/N_2 \rceil}
%A_{m_1\; \mathrm{mod}\; N_1, m_2\; \mathrm{mod}\; N_2} \mathrm{vec}(X)_{m_2} \\ 
%=&\sum_{m_2=1}^{N_2 N_3} B_{\lceil m_2/N_2 \rceil, \lceil m_1/N_1 \rceil}
%A_{m_1\; \mathrm{mod}\; N_1, m_2\; \mathrm{mod}\; N_2} X_{m_2\; \mathrm{mod}\; N_2, \lceil m_2/N_2 \rceil} \\ \tag{5}
%\end{align}$$ 
%
%It would seem that we are close, but it is not obvious that (4) and (5) are equal.
%Although $A$'s row index $m_1\; \mathrm{mod}\; N_1$ and $B$'s column index $\lceil m_1/N_1 \rceil$ agree, the other do not.
%It is suggestive that maybe if we "sum away" $n_2, n_3$, we might find them to agree, but $A,X,B$ are all free, and we have nothing to exploit to carry out that summation.

% % % % % % % % % % % % % % % % % % % % 

\Result
{Definition.}
{
Define \m {\Mr {vec} \SB {\M {A}} \in \Mb {V}_{\Mb {K}} \SB {N_1, N_2}} to be the vectorization of \m {\M {A} \in \Mb {M}_{\Mb {K}} \SB {N_1, N_2}}.
Formally,
\Disp{
\NC \RB {\Mr{vec} \SB {\M{A}}} \DB {m_1}
=\NC \M{A} \DB {m_1\; \Mr{Mod}\; N_1, \Fl{ m_1/N_1 }} \NR[+]
}
}

\Result
{Definition.}
{
For \m {\M{A} \in \Mb {M}_{\Mb {K}} \SB {N_1, N_2}} and \m {\M{B} \in \Mb {M}_{\Mb {K}} \SB {N_3, N_4}}, define the Kronecker product \m {\M{A} \otimes \M{B} \in \Mb {M}_{\Mb {K}} \SB {N_1 N_3, N_2 N_4}} by
\Disp{
\NC \NC \RB {\M{A} \otimes \M{B}} \DB {m_1, m_2} \NR
\NC =\NC \M{A} \DB {\Fl{ m_1/N_3 }, \Fl{ m_2/N_4 }}
\M{B} \DB {m_1\; \Mr{Mod}\; N_3, m_2\; \Mr{Mod}\; N_4} \NR[+]
}
}

With (), scholars often exploit the relation

\Result
{Lemma.}
{
Suppose \m {\M{A} \in \Mb {M}_{\Mb {K}} \SB {N_1, N_2}, \M{X} \in \Mb {M}_{\Mb {K}} \SB {N_2, N_3}, \M{B} \in \Mb {M}_{\Mb {K}} \SB {N_3, N_4}},
Then
\Disp {
\NC \mathrm{vec} \RB {\M{A}\M{X}\M{B}}
= \NC \RB {\M{B}^\Tr \otimes \M{A}} \mathrm{vec} \SB {\M{X}} \NR
}
}

The proof is not particularly easy to find, and we show it here.
We thus verify that arbitrary components of lhs and rhs agree.
On one hand,

\Disp{
\NC \NC \RB {\Mr{vec} \SB {\M{A}\M{X}\M{B}}} \DB {m_1} \NR
\NC = \NC \RB {\M{A}\M{X}\M{B}} \DB {m_1\; \Mr{Mod}\; N_1, \Fl{ m_1/N_1 }} \NR
\NC = \NC \sum _{n_2=1} ^{N_2} \sum _{n_3=1} ^{N_3}
\M{A} \DB {m_1\; \Mr{Mod}\; N_1, n_2} \M{X} \DB {n_2, n_3} \M{B} \DB {n_3, \Fl{ m_1/N_1 }} \NR[+]
}

On the other hand,

\Disp{
\NC \NC \RB {\RB {\M{B}^\Tr \otimes \M{A}} \Mr{vec} \SB{\M{X}} } \DB {m_1} \NR
\NC = \NC \sum _{m_2=1} ^{N_2 N_3} \RB {\M{B}^\Tr \otimes \M{A}} \DB {m_1, m_2} \Mr{vec} \SB {\M{X}} \DB {m_2} \NR
\NC = \NC \sum _{m_2=1} ^{N_2 N_3} \RB {\M{B}^\Tr} \DB {\Fl{ m_1/N_1 }, \Fl{ m_2/N_2 }}
\M{A} \DB {m_1\; \Mr{Mod}\; N_1, m_2\; \Mr{Mod}\; N_2} \Mr{vec} \SB {\M{X}} \DB {m_2} \NR 
\NC = \NC \sum _{m_2=1} ^{N_2 N_3} \M{B} \DB {\Fl{ m_2/N_2 }, \Fl{ m_1/N_1 }}
\M{A} \DB {m_1\; \Mr{Mod}\; N_1, m_2\; \Mr{Mod}\; N_2} \M{X} \DB {m_2\; \Mr{Mod}\; N_2, \Fl{ m_2/N_2 }} \NR[+]
}

In () and (), the row index of \m {\M{A}} and \m {\M{B}} already agree.
The key observation is that when \m {n_2} runs through \m {0, \dots, N_2-1} and \m {n_3} runs through \m {0, \dots, N_3-1}, it follows \m {n_2 n_3} runs through \m {0, \dots, N_2 N_3-1}.
We see two expressions are exactly the same, and we are done.

The matrix consisting of the collection of possible response according to available paths --- called the dictionary --- plays the role of the sensing matrix of OMP.
The combiner at the receiver end corresponds to the step where we map the data from high to low dimension.

\stopsection
\startsection [title={Further Development on Compressive Channel Sensing}]

The literature on Compressive Channel Sensing has since been vast, and we need not and cannot go through all of them here.
Rather, we analyze the respective contribution and weakness on some of the recent seminal papers.

Rao and Lau (2014) considers a distributed algorithm for a MU-MIMO and proposes a modified joint OMP algorithm, with the assumption on the sparsity of the channel and that channels seen by various users shall share the same support.

Alkhateeb et.\ al.\ (2014) suggests an adaptive OMP algorithm, using a (beam) code book with quantized AoA and AoD.

Alkhateeb, Leus, and Heath Jr.\ (2015) poses the interesting trade-off of number of OMP measurement and accuracy in an application of OMP.
They work simplified all-phase-shifter combiner model, where every receiver antenna may only multiply the signal by a constant, and the channel model is assumed to take value on a quantized, non-uniform set of angles.
Their work mainly focuses on numerical simulation, rather than mathematical proof.

Since the notice of hybrid beamforming, scholars have also applied the same idea to hybrid systems.
Lee, Gil, and Lee (2016) considers a hybrid beamforming system, where the composition of the precoders and combineres play the role of sensing matrix.
The beambook phase is quantized, which may have been a source of error, and they have specifed a particular grid set of angles and argues the superiority.
One of their simulations experiments utilizes DFT and permutated DFT matrix, instead of as originally suggested by Tropp and Gilbert (2007).

Meanwhile, the improvement of OMP bound is also going on.
Cai, Wang, and Xu (2010), making assumptions on MIP matrices, give new bound on OMP, and, in addition, Cai and Wang (2011) replicates the same for analysis for DS, among other convex algorithms.
Indeed, MIP deals with correlation of columns of the sensing matrix, rather than the almost-unitarity, making the condition easier to verify.

Ben-Haim et.\ al.\ (2010) follow up their work and refine their bounds, also considering MIP conditions.
They conclude that OMP is better for low-SNR scenario, and DS is better for high-SNR.
Still, since the setting of OMP and DS is rather different, it remains unclear which of is better.

Alkhateeb, Leus, and Heath (2015), being a recent paper, cites Ben-Haim et.\ al.\ (2010) as the main foundation for OMP perfomance bound.

Gao et.\ al.\ (2015) discusses the jointly reconstruction of several high-dimensional sparse signals having the same support, using different measurement matrices in a modified basis pursuit problem.

We remark several observation on past literature, that has been, to the best of our knowledge, lacking or less than ideal.

\startitemize
\item Previous work favors OMP rather than DS, let alone other sparse algorithm.
\item Previous work simply assumes the bound of the norm of the channel matrix.
\item The requirement that the sensing matrix should be elementwise i.i.d.\ Gaussian (Tropp and Gilvert 2007) seems to be more restrictive than RIP (Cand\`es and Tao 2007).
\item The quantization of angle in generating may be a problem, and that is difficult to analyze in OMP's setting
\item No bounds is derived that explicitly depends on the sparsity of the channel parameters as, say, number of channel paths and array response.
\stopitemize

\startsection [title={Contribution}]

We consider a system model where both digital and analog beamformer are i.i.d.\ random matrix combiner that serve readily as the sensing matrix satisfying the RIP property.
We apply a convex program analogous of DS in order to directly estimate the channel under hypothesis of uniform linear array response.
Moreover, inspired by the lines of proof of Cand\`es and Tao (2006), we give a bound that indicates our method is to be successful for overwhelming probability.

Our contribution includes:

\startitemize
\item As a {\it prima face} difference, we use a modified DS rather than OMP, done on the beamspace rather than the spatial domain, and involving complex numbers rather than real numbers.
\item Moreover, an explicit bound has been shown to provide realistic guide for engineering, where the sparsity of the virtual channel matrix, rather than just assumed, is written out as a function of the number of paths of the channel.
\item Accordingly, the concern of quantization error of the virtual channel's phase angle (which is inevitable when designing the OMP beambook) has already been considered in the proof of our main bound.
\item We derive explicitly the SOCP problem and afterwards its dual problem, and wrote a proof-of-concept but efficient code.
\stopitemize

\stopsection

\stopchapter

