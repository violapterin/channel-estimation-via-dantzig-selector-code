\startsection [title={Almost-Sparsity of Angular Channel Response}]
\startsubsection [title={Norm of Array Response}]

Our plan is first to show that \m {\SB {\V {g}}} is almost sparse, and substitute the bound into the expected square error of DS, thus generalizing the original argument.
We will continue to use the variables \m {\V {P}}, \m {\V {g}}, \m {\V {y}}, and \m {\V {z}} introduced in previous sections.

Split \m {\V{g}} into two parts: \m {\V {g} _\MC {A}} is the largest \m {S} positions of \m {\V{g}}, \m {\V {g} _\MC {A}} the next \m {S} largest positions of \m {\V{g}}, and \m {\V {g} _\MC {K}} are the positions complement to \m {\V {g} _\MC {A}}.
For example, if \m {\V {g} =\IP {-1,3,-4,2,8}}, and \m {S=2}, then \m {\V {g} _\MC {A} =\IP {0,3,0,0,8}}, \m {\V {g} _\MC {B} =\IP {-1,0,0,2,0}}, and \m {\V {g} _\MC {A} =\IP {-1,0,-4,2,0}}.
We rephrase it formally.

\Result
{Definition}
{
Consider \m {\V{x} \in \MB {V}_{\MB {K}} \SB {N_p}}.
Let bijection \m {\s: \MC {T} \SB {N_p} \mapsto \MC {T} \SB {N_p}} be defined to sort \m {\V{x}} in decreasing magnitude, meaning
\Disp{
\NC \V{v} \DB {\s \SB {n}}
\leq \NC \V{v} \DB {\s \SB {n-1}} \NR[+]
\NC n =\NC 1, \ldots, N_p-1 \NR
}
Its existence is evident.

For fixed \m {S}, define
\Disp{
\NC \MC {S} \SB {\V{x}, S}
=\NC \sum_{i=0} ^{S-1} \V {x} \DB {\s \SB{i}} \V {u} _{\s \SB{i}} \NR[+]
\NC \MC {C} \SB {\V{x}, S}
=\NC \sum_{i=S} ^{N_p} \V {x} \DB {\s \SB{i}} \V {u} _{\s \SB{i}} \NR[+]
}
}

Now set
\Disp{
\NC \V{g} _{\MC {A}}
=\NC \MC {S} \SB {\V{g}, S} \NR[+]
\NC \V{g} _{\MC {K}}
=\NC \MC {C} \SB {\V{g}, S} \NR[+]
\NC \V{g} _{\MC {B}}
=\NC \MC {S} \SB {\V{g} _\MC {K}, S} \NR[+]
}
We hope that \m {\VNm {\V{g} _\MC {K}}_1} is small.

\Result
{Definition}
{
For \m {\V {x} \in \MB {V}_{\MB {K}} \SB {N}}, we say \m {\V {x}} is almost-\m {s}-sparse according to the \m {\ell_p}-norm with remainder \m {R}, if
\Disp {
\NC \VNm {\MC {C} \SB{\V {x}, s}} _p
\leq \NC R. \NR[+]
}
}

Let \m {\f} be fixed.
It suffices to bound
\Disp {
\NC \NC \VNm {
   \MC {C} \SB {\M {K}^\Adj \V {a} \SB {\f}, s}
} _2 ^2 \NR[+]
}

Towards that end, introduce
\Disp {
\NC \psi \SB {\f, n_H}
:=\NC \RB {
   \RB {
      \RB {\f \; \Rm{Mod}\; \F{2\pi}{N_H}}
      +\F {2 \pi n_H} {N_H}
      +\pi
   } \;
   \Rm{Mod}\; \RB {2\pi}
}
-\pi \NR
}
Note that by construction
\Disp{
\NC \Nm {\psi \SB {\f, n_H}}
\leq \NC \pi \NR
}
And define the so-called Dirichlet kernel
\Disp {
\NC D \SB {\psi'}
:= \NC \sum_{n_H=0}^{N_H-1} \Ss {e}^{i n_H \psi'} \NR
}
Then observe
\Disp {
\NC \RB {\M {K}^\Adj \V {a} \SB {\f}} \DB {n_H}
=\NC \F {1}{N_H} D \SB {\psi \SB {\f, n_H}} \NR
}

Now, it can be verified that
\Result
{Lemma}
{
For \m {-\pi \leq x < \pi}, we have
\Disp {
\NC \Nm {x - \F {x^3}{6}} \leq \NC \Nm {\sin x}. \NR[+]
}
}

Applying Lemma () to the denominator of () and bounding the nominator by 1, we have
\Disp {
\NC \Nm {D \SB {\psi'}}
= \NC \F {\Nm {\sin \SB {N_H \psi'/2}}}{\Nm {\sin \SB {\psi' /2}}} \NR
\NC \leq \NC B \SB {\psi'} \NR
\NC := \NC \F {48}{\Nm {\psi'^2 -24} \Nm {\psi'}} \NR[+]
\NC -\pi \leq \NC \psi' < \pi. \NR
}
Thus,
\Disp {
\NC R^2
=\NC \F {1}{N_H^2}
\VNm {
\MC {C}
\SB{
   \sum_{n_H' =0}^{N_H -1}
   B \SB {\psi \SB {\f, n_H'}}
   \V {u}_{n_H'},
   s
}
} _2^2
\NR[+]
}
Note that \m {\Nm {B \SB {\psi'}}} is strictly decreasing in \m {\SB {0,\pi}}.
We seek to bound the \quotation {rectangulars} with an integral, and we have to split the cases that \m {N_H} is odd and even.
Anyway, a moment's reflection shows
\Disp {
\NC R^2
\leq \NC \F {1}{N_H^2} \D \F {N_H}{2\pi} \D 2 \int_{\pi s/N_H}^{\pi} B \SB {\psi'} ^2 d \psi' \NR
\NC = \NC \F {2304} {N_H \pi^6}
\int _{s /N_H} ^1 \F{1} {(24/\pi^2 -x'^2)^2 x'^2} dx' \NR
\NC =\NC \F{1} {2\pi^2 N_H}
\RB {
  -\F {8} {u}
+\F {4\pi^2 u} {24 -\pi^2 u^2}
+\R{6} \pi \tanh^{-1} \SB {\F {\pi u} {2\R{6}}}
}
\Bigg \| _{s/N_H} ^1 \NR
\NC \leq \NC \F {4} {\pi^2} \D \F {1} {s} +\F {0.032670} {N_H} \NR[+]
}
In the last step we bound the lower limit of the second and third term with \m {x' =0}, and leave the dominating \m {1/x'}.

Expanding \m {R} gives
\Disp {
\NC R
\leq \NC \F {2} {\pi} \D \F {1} {\R{s}} +\F {\R {s}} {38 N_H}. \NR[+]
}

Fixing \m {N_H}, it shall be easy to see that rhs is strictly decreasing in \m {N_H}, and the value of \m {s} that minimizes it is about \m {s =5N_H}.
In the range that we care, rhs is thus strictly decreasing in \m {s}.
\m {N_H}.

\Result
{Lemma}
{
Let \m {\f \SB {\o_h}} be given, and linear array response \m {\V {a} \SB {\f}} defined as in ().
Suppose \m {N_H \geq 4}.
Then, for all instances of \m {\f}, \m {\V {a} \SB {\f}} is almost-\m {s}-sparse according to the \m {\ell_2}-norm with remainder \m {R} to be
\Disp{
\NC R
\leq \NC \m {\F{1}{3 \R{N_H}}} \NR[+]
}
}

Since the bound is also decreasing in \m {N_H}, for concreteness, let us plug in \m {N_H =4} for the upper bound, giving
\Disp{
\NC R
\leq \NC \F {1} {6} \NR[+]
}

\stopsubsection

\startsubsection [title={Norm of Angular Channel Response}]

It remains to bound \m {\VNm {\M {G}} _F}.
Note that
\Disp{
\NC \VNm {\M {G}}_F
=\NC \VNm {\M {H}}_F \NR[+]
}
So it suffices to consider \m {\VNm {\M {H}} _F}.

\Disp{
\NC \VNm {\sum _{l=0} ^{L-1} \a_l \V {a} \SB {\f_l} \V {a} \SB {\th_l} ^\Adj } _F
\leq \NC \sum _{l=0} ^{L-1} \Nm {\a_l}\VNm {\V {a} \SB {\f_l} \V {a} \SB {\th_l} ^\Adj } _F \NR
\NC \leq \NC
\sum _{l=0} ^{L-1}
\Nm {\a_l} \VNm {\V {a} \SB {\f_l}} _2
\VNm {\V {a} \SB {\th_l}} _2 \NR[+]
}
Lastly, we know \m {\ell_1}-norm is larger than \m {\ell_2}-norm.
Since \m {\ell_1}-norm will be used in later proof, we display the statement about \m {\ell_1}-norm.
\Disp{
\NC \NC \MB {E} \SB {\VNm {\MC {C} \SB {\sum _{l=0} ^{L-1} \a_l \V {a} \SB {\f_l} \V {a} \SB {\th_l} ^\Adj}} _F} \NR
\NC \leq \NC \F {1} {9 \R{N_H}} \MB {E} \SB {\sum _{l=0} ^{L-1} \Nm {\a_l}} \NR
\NC \leq \NC \F {\R{\pi} L} {3 \R{N_H}} \NR[+]
}

\Result
{Lemma}
{
Let \m {\f \SB {\o_h}} and \m {\th \SB {\o_h}} be uniformly, independently distributed in \m {[0,2\pi)}, and let \m {\V {g} \in \MB{V}_\MB{C} \SB{N_h}} be defined as in ().
If \m {N_h \geq 4^2 =16}, then \m {\V {g}} is almost-\m {s^2 L}-sparse with \m {\ell_1}-residue \m {R} to be
\Disp {
\NC R
\leq \NC \F {\R{\pi} L} {3 \R{N_H}} \NR
}
}

\stopsubsection
\stopsection

