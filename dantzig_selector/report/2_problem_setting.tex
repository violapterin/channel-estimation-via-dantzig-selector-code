quad\startchapter [title={Problem Setting}]

\startsection [title={Channel Model}]

Since expression involving probability will get complicated, sometimes we explicitly mark the independence on event space index.

Conceptually, we use \m {\O_t} to denote the event space of the transmitter end, \m {\O_h} to denote that of the communication channel itself, \m {\O_z} to denote that of the noise arising in the communication channel, and \m {\O_r} to denote that of the receiver end.
It helps to think \m {\NC \O = \NC \O_t \times \O_h \times \O_n \times \O_r \NR} as the overall event space.

We restrict our consideration to the uniform linear array, which can be modeled as
\Disp {
\NC \V {a} \SB {\psi'}
\in \NC \MB {V}_\MB {C} \SB {N_H} \NR
\NC \V {a} \SB {\psi'}
= \NC \F {1}{\R {N_H}} \sum_{n_h=0}^{N_H-1} \Ss {e} ^{\Ss {i} n_h \psi'} \V {u}_{n_h} \NR
}
where \m {\Ss {e}} is the base of natural logarithm, and \m {\Ss {i}} the imaginary unit.

And consider the virtual representation of the MIMO channel (see for example Akdeniz et.\ al.\ 2014).
\Disp {
\NC \M{H}
=\NC \sum_{l=0} ^{L-1}
\a_l
\V {a} \SB { 2\pi \F {d_{\Rm {arr}}} {\l _{\Rm {arr}}} \sin \f_l'}
\V {a} \SB { 2\pi \F {d_{\Rm {arr}}} {\l _{\Rm {arr}}} \sin \th_l'}^\Adj \NR[+]
}
The physical meaning of \m {\th_l} is the \m {l}-th angle of incidence (formed by the ray and the normal line) of departure electronic wave, and \m {\th_l}, the \m {l}-th that of arrival wave, while \m {d_{\Rm {arr}}} is the distance between two adjacent antennae.
For our purpose, we may absorb the argument of \m {\V {a}} as
\Disp {
\NC \f_l
= \NC \RB {
  2\pi \F {d_{\Rm {arr}}} {\l_{\Rm {arr}}} \sin \f_l'
  \; \Rm{Mod}\; \RB {2\pi}
} \NR[+]
\NC \th_l
= \NC \RB {
  2\pi \F {d_{\Rm {arr}}} {\l_{\Rm {arr}}} \sin \th_l'
  \; \Rm{Mod}\; \RB {2\pi}
} \NR[+]
}
to get a simpler form
\Disp {
\NC \M{H}
\in \NC \MB {M}_\MB {C} \SB {N_H} \NR
\NC \M{H}
=\NC \sum_{l=0} ^{L-1} \a_l \V {a} \SB {\f_l} \V {a} \SB {\th_l}^\Adj \NR
}

\stopsection

\startsection [title={System Model}]

We consider the hybrid configuration at both transmitter and receiver end.
That is to say, each end consists of both digital and analog percoders and combiners, as follows.
In the transmitter end, there are, in order seeing from the end, digital precoder \m {F_B} and analog precoder \m {F_R}.
Similarly, in the receiver end, there are, in order seeing from the end, digital combiner \m {W_B} and analog combiner \m {W_R}.
To set up notation, we list them as below.
\Disp {
\NC \M {F}_B \in \NC \MB {M}_{\MB {C}} \SB {N_R, N_Y} \NR
\NC \M {F}_R \in \NC \MB {M}_{\MB {C}} \SB {N_H, N_R} \NR
\NC \M {W}_R \in \NC \MB {M}_{\MB {C}} \SB {N_R, N_H} \NR
\NC \M {W}_B \in \NC \MB {M}_{\MB {C}} \SB {N_Y, N_R} \NR
}
Recall that analog precoder is restricted to have value with magnitude being unity, i.e.,
\Disp {
\NC \Nm {\M {F}_R \DB {n_H, n_R}}
= \NC 1, \NR[+]
\NC \Nm {\M {W}_R \DB {n_R, n_H}}
= \NC 1, \NR[+]
\NC n_H
= \NC 0, \dots N_H-1, \NR
\NC n_R
= \NC 0, \dots N_R-1 \NR
}
And we assume
\Disp {
\NC N_H \gg \NC N_R \NR[+]
\NC N_R \gg \NC N_Y \NR[+]
}

This is analogous to (), and we will exploit the fact when applying compressive sensing techinique.

We consider zero-mean, unity-variance additive white Gaussian noise (AWGN).
Since we restrict our discussion to a specific slice of time, this random process may be simply taken to be a matrix \m {\M{Z}} with each entry being i.i.d.\ standard Gaussian.
\Disp{
\NC \M {Z} \in \NC \MB {M}_{\MB {C}} \SB {N_Y, N_Y} \NR[+]
}
We may introduce the effective channel
\Disp {
\NC \M{Y}
:=\M{W}_B \M{W}_R \RB {\M{H} \M{F}_R \M{F}_B +\M{Z}} \NR
}
We restate the equation with event space index for greater clarity,
\Disp {
\NC \NC \M{Y} \SB {\o_t, \o_h, \o_z, \o_r} \NR
\NC := \NC \M{W}_B \SB {\o_r} \M{W}_R \SB {\o_r} \RB {\M{H} \SB {\o_h} \M{F}_R \SB {\o_t} \M{F}_B \SB {\o_t} +\M{Z} \SB {\o_z} } \NR
}

Our task, then, is the recovery of \m {\M {H}}, and the design of \m {\M {W}_R}, \m {\M {W}_B}, \m {\M {F}_R}, and \m {\M {F}_B}.

\stopsection
\startsubsection [title={Vectorization}]

First, for short we set
\Disp {
\NC N_h
= \NC N_H^2 \NR[+]
\NC N_y
= \NC N_Y^2 \NR[+]
}
If we had followed the widely accepted approach of compressive channel sensing, we would have written
\Disp {
\NC \V {h}
\in \NC \MB {V}_{\MB {C}} \SB {N_h} \NR
\NC \V {h}
:= \NC \Rm {vec} \SB {\M {H}} \NR[+]
\NC \V {y}
\in \NC \MB {V}_{\MB {C}} \SB {N_y} \NR
\NC \V {y}
:= \NC \Rm {vec} \SB {\M {Y}} \NR[+]
\NC \V {z}
\in \NC \MB {V}_{\MB {C}} \SB {N_y} \NR
\NC \V {z}
:= \NC \Rm {vec} \SB {\M{W}_B \M{W}_R \M {Z}} \NR[+]
\NC \M {Q}
\in \NC \MB {M}_{\MB {C}} \SB {N_y, N_h} \NR
\NC \M {Q}
:= \NC \RB {\M {F}_B^\Tr \M {F}_R^\Tr} \otimes \RB {\M {W}_B \M {W}_R} \NR[+]
}
to formulate our goal as sparse recovery problem of \m {\Rm {vec} \SB {\M{H}}} to be
\Disp {
\NC \V {y}
=\M {Q} \V {h} +\V {z} \NR
}
As we have motivated in previous sections, we want to use DS by considering sparse vector, and at the same time to get rid of the angle grid quantization problem.
However, there is no guarantee, though, that \m {\V {h}} will be sparse.

\stopsection
\startsection [title={Working in the Angular Space}]

Introduce the discrete Fourier transform matrix \m {\M {K}} defined to be
\Disp {
\NC \M {K} \in  \NC \MB {M}_{\MB {C}} \SB {N_H, N_H} \NR
\NC \M {K} \DB {n_1, n_2}
= \NC \F {1} {\R {N_H}} \Ss {e}^{2\pi \Ss{i} n_1 n_2 /N_H}, \NR[+]
\NC \Q n_1, n_2
= \NC 0, 1, \dots, N_H-1 \NR
}
Roman \m{\Ss{e}} and \m{\Ss{i}} denote the natural base and imaginary unit, respectively.
Here recall that we have
\Disp{
\NC \M {K}^\Adj \M {K}
= \NC \M {I} \NR[+]
}
Observe that, if we write
\Disp {
\NC \M {G}
=\NC \M {K}^\Adj \M {H} \M {K} \NR[+]
}
which has the interpretation as the angular space (i.e., spatial frequency domain) representation of \m {\M {H}}, then
\Disp {
\NC \M {Y}
\in \NC \MB {M}_{\MB {C}} \SB {N_Y, N_Y} \NR
\NC \M {Y}
=\NC \M {W}_B \M {W}_R \M {K} \cdot \M {G} \cdot \M {K}^\Adj \M {F}_R \M {F}_B
+\M {W}_B \M {W}_R \M {Z} \NR[+]
}
If at the end of day, \m {\M {G}} is sparse, then we recover \m {\M {G}} instead, and our use of DS is fully justified.

That is to say, a similar Kronecker product may be used.
To simplify notation, set
\Disp {
\NC \M {P}
\in \NC \MB {M}_{\MB {C}} \SB {N_y, N_h} \NR
\NC \M {P}
:=\NC \RB {\M {F}_B^\Tr \M {F}_R^\Tr \M{K}^\ast} \otimes \RB {\M {W}_B \M {W}_R \M {K}} \NR[+]
}
And accordingly
\Disp {
\NC \V {g}
\in \NC \MB {M}_{\MB {C}} \SB {N_h} \NR
\NC \V {g}
:= \NC \Rm {vec} \SB {\M {G}} \NR[+]
}
Then
\Disp {
\NC \V {y}
=\NC \M {P} \V {g} +\V {z} \NR[+]
}

\stopsection
\startsection [title={Proposed Method}]

If so, our program reads

\Result
{Algorithm}
{
\startitemize[n]
\item Input \m{\M {P} \in \MB {M}_{\MB {C}} \SB {N_y, N_h}}, \m{\V {y} \in \MB {V}_{\MB {C}} \SB {N_h}}, \m{\g \geq 0}.
\item Calculate
\Disp {
\NC \Hat {\V {g}}
\LA \NC \startcases
\NC \Min {\V {g}' \in \MB {V}_{\MB {C}} \SB {N_h}}  \MC \VNm {\V {g}'} _1 \NR[+]
\NC \Rm {subject} \; \Rm {to} \Q \MC \VNm {\M {P}^\Adj \RB {\V {y} -\M {P} \V {g}'}} _\infty \leq \g \NR
\stopcases \NR
}
\item Convert \m {\Hat {\V {g}}} back to the space domain, namely
\Disp {
\NC \Hat {G}
\LA \Rm {vec}^{-1} \SB {\Hat {g}} \NR
}
\item Recover the estimated \m {\Hat {\M {H}}}, as
\Disp {
\NC \Hat {\M {H}}
\LA \M {K} \Hat {\M {G}} \M {K}^\Adj \NR
}
\item Output \m {\Hat {\M {H}}}.
\stopitemize
}

It, then, will not be hard to establish Cand\`es and Tao's result to the setting of millimeter wave virtual channel, and resulting bound is immediate under straighforward work, as follows.

\stopsection
\stopchapter
