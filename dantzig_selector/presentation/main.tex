% TODO:
% Larger font (e.g. 14pt), Larger line spacing
% Chinese font? Font mixing mechanism?
% `\notag` equivalent: A wrapper class?
% Table of contents: wider vertical spacing? Font?
% Book title: wider line spacing?
% Chapter 0
% Table width

% % % % % % % % % % % % % % % % % % % % % % % % % % % % % % % %

% XXX
% Hold regex:
% Replace math mode:
%       %s/\\(\(.\{-}\)\\)/\\m{\1}/g
% Reformatting:
%       %s/\(\a\){/\1 {/g
%       %s/}\([\\(\[]\)/} \1/g
%       %s/\(\a\) }/\1}/g
%       %s/{ \(\a\)/{\1/g
%       %s/\([}\a]\)_/\1 _/g
%       %s/(\(.\{-}\))/\\RB{\1}/gc
%       %s/(\(.\{-}\))/\\SB{\1}/gc

% % % % % % % % % % % % % % % % % % % % % % % % % % % % % % % %

\define [0] {\a} {\alpha}
\define [0] {\g} {\gamma}
\define [0] {\d} {\delta}
\define [0] {\e} {\epsilon}
\define [0] {\f} {\varphi}
\define [0] {\l} {\lambda}
\define [0] {\s} {\sigma}
\define [0] {\i} {\iota}
\define [0] {\th} {\vartheta}
\define [0] {\O} {\Omega}
\define [0] {\o} {\omega}
\define [0] {\D} {\cdot}
\define [0] {\Adj} {\dagger}
\define [0] {\Tr} {\intercal}
\define [0] {\LA} {\leftarrow}
\define [0] {\Q} {\quad}
\define [0] {\FourQ} {\Q \Q \Q \Q}
\define [0] {\SixQ} {\Q \Q \Q \Q \Q \Q}

\define [0] {\AuthorText}
{
   Author: Tzu-Yu Jeng \\
   Advisor: Prof.\ H.\ J.\ Su \\
   Nov.\ 2018 to (Mar.\ 2019) \_\_\_\_
}
\define [0] {\TitleText}
{
   Compressive Channel Sensing \\
   of Multipath Millimeter Channel \\
   via an Angular Domain Dantzig Selector \\
   and its Probabilistic Performance Bound
}
\define [0] {\TheCount}
{
   \incrementcounter [NumResult]
   {\bf \rawcountervalue [NumResult]}
}

\define [1] {\MB} {{\mathblackboard {#1}}}
\define [1] {\MC} {{\mathscript {#1}}}
\define [1] {\MF} {{\mathfraktur {#1}}}
\define [1] {\Rm} {{\rm {#1}}}
\define [1] {\It} {{\it {#1}}}
\define [1] {\Bf} {{\bf {#1}}}
\define [1] {\Ss} {{\ss {#1}}}
\define [1] {\RB} {\left( #1 \right)}
\define [1] {\SB} {\left[ #1 \right]}
\define [1] {\CB} {\left{ #1 \right}}
\define [1] {\DB} {\left[ \! \left[ #1 \right] \! \right]}
\define [1] {\Fl} {\left \lfloor #1 \right \rfloor}
\define [1] {\Cl} {\left \lfloor #1 \right \rfloor}
\define [1] {\Nm} {\left \vert #1 \right \vert}
\define [1] {\VNm} {\left \Vert #1 \right \Vert}
\define [1] {\R} {\sqrt {#1}}
\define [1] {\Min} {\underset {#1} {\Rm {min}}}
\define [1] {\IP} {\left \langle #1 \right \rangle}
\define [1] {\V} {\underline {#1 \mkern -1mu} \mkern 2mu}
\define [1] {\T} {\tilde {\vphantom {\overline{#1}} #1}}
\define [1] {\Hat} {\hat {\vphantom {\overline{#1}} #1}}
\define [1] {\Stack} {\startsubstack #1 \stopsubstack}

\define [1] {\Disp}
{
   \placeformula
   \startsplitformula
   \startalign
      #1
   \stopalign
   \stopsplitformula
}
\define [1] {\M}
{
   \underline {{\underline {#1 \mkern -1mu} \mkern 2mu} \mkern -2mu}
   \mkern 2mu
}
\define[1] {\Uurrll} {\goto {\mono {\hyphenatedurl {#1}}} [url (#1)]}

\define [2] {\F} {\frac {#1}{#2}}

\define[2] {\Result}
{
   \bigskip
   \noindent \hskip 3em
   \TheCount \; {\bf #1} \par
   #2 \par
   \hfill
   \m{\blacksquare} \par
}

% % % % % % % % % % % % % % % % % % % % % % % % % % % % % % % %

\definemathmatrix[TheMatrix][left={\left[}, right={\right]}]
\definecounter [NumResult] [start=0, way=text]
\definefontfeature [default] [default] [onum=yes]
\definefontfamily [MainFace] [rm] [TeX Gyre Pagella]
\definefontfamily [MainFace] [ss] [TeX Gyre Heros]
\definefontfamily [MainFace] [tt] [TeX Gyre Cursor] [features=none]
\definefallbackfamily [MainFace] [mm] [TeX Gyre DejaVu Math] [force=yes,range=uppercasescript]
\definefallbackfamily [MainFace] [mm] [TeX Gyre DejaVu Math] [force=yes,range=lowercasescript]
\definefontfamily [MainFace] [mm] [TeX Gyre Pagella Math]
\definemeasure [pointoneline] [1.5\baselineskip]
\definecolor [headingcolor] [r=0.2, g=0.25, b=0.5]
\definealternativestyle [TitleStyle] [\ss\bfd] []
\definealternativestyle [AuthorStyle] [\ss\tfb] []
\definemakeup [titlepage] [align=middle]

% % % % % % % % % % % % % % % % % % % % % % % % % % % % % % % %

\setupparagraphs [n=3, before={\blank}, after={\blank}]
\setupbodyfont [MainFace, 12pt]
\setupinteraction [state=start]
\setupcolors [state=start]
\setuppapersize [A4] [A4]
\setuppagenumbering [location={footer,middle}, style=slanted]
\setuphead [chapter, section, subsection] [color=headingcolor]
\setuphead [chapter] [style={\ss\bfd}]
\setuphead [section] [style={\ss\tfc}]
\setuphead [subsection] [style={\ss\itb}]
\setuphead[title] [incrementnumber=list, number=no]
\setuphead[subject] [incrementnumber=list, number=no]
\setuphead[subsubject] [incrementnumber=list, number=no]
\setuplist[chapter] [width=2cm]
\setuplist[section] [width=2cm]
\setuplist[subject] [width=2cm]
\setupitemize [inbetween={}, style=bold]
\setuptype [lines=hyphenated]
\setupindenting[big,always]
\setupwhitespace[medium]
\setupcombinedlist [content] [list={chapter, title, section, subject, subsection, subsubject}]

\setuplayout [
   backspace=17mm,
   margin=15mm,
   margindistance=0mm,
   width=168mm,
   topspace=13mm,
   height=260mm,
   header=10mm,
   footer=10mm,
   bottomdistance=0mm,
   bottom=8mm
]




\starttext


\blank [big, force]

\Title {Dantzig Selector \\ Applied on mm-Wave MIMO: \\ Part I. Expected Error Analysis}
\blank [big]

\Subtitle {Presenter: Tzu-Yu Jeng \\ Advisor: Prof.\ H.J.\ Su}
\blank [big]

\Subsubtitle{May 30, 2019}
\Subsubtitle{Graduate Instute of Commnication Engineering, \\ National Taiwan University}
\page [yes] % % % % % % % % % % % % % % % % % % % % % % % % % %
\Frame {Organization}
{
\I Problem Configuration Proposed Method

\I Sparsity of Angular Channel Matrix

\I Concentration Inequality of \m {\M{P}}

\I Restricted Isometry of \m {\M{P}}

\I Expected Error of Dantzig Selector

\I Conclusion and Future Work

\I References
}
\page [yes] % % % % % % % % % % % % % % % % % % % % % % % % % %
\Frame {Background}
{
\I Multiple-input-multiple-output (MIMO) wireless communication is expected to be the next-generation r\'egime

\I But channel response has to be known at the receiver to facilitate applications like beamforming and channel calibration

\I And in the MIMO system, estimating wireless channels calls for high complexity

\I Meanwhile, in the millimeter wave r\`egime, which is often used together with MIMO, channel often exhibits sparse properties

\I If the sparsity is exploited, can few observations suffice to estimate the channel?
}
\page [yes] % % % % % % % % % % % % % % % % % % % % % % % % % %
\Frame {Compressive Sensing}
{
\I Such case of channel estimation is facilitated by advances of so-called compressed sensing (CS)

\I CS addresses the situation that the number of model parameters \m {N_p} the number of measurements \m {N_m}.

\I With insufficient (possibly noisy) measurements, under what circumstances can we recover all \m {N_p} parameters?

\I Work on CS reveals that few observations of the signal may be sufficient for us to resonstruct the signal when it is sparse.

\I In the advent, Cand\`es and Tao (2006) proposes the Dantzig Selector (DS) as an possible answer.
}
\page [yes] % % % % % % % % % % % % % % % % % % % % % % % % % %
\Frametitle {Notation}

{\tfx
\starttable[|l|l|l|] \HL
\NC Expression \VL Meaning or Definition \VL More Explanation \SR \HL
\NC \m {f \SB {x}} \VL square brackets \VL only for arguments \AR \HL
\NC \m {\V {a}} \VL vector \VL  \AR \HL
\NC \m {\M {A}} \VL matrix \VL  \AR \HL
\NC \m {\V {a} \DB {n}} \VL \m {n}-th component of \m {\V {a}} \VL  \AR \HL
\NC \m {\M {A} \DB {m,n}} \VL \m {m,n}-th entry of \m {\M {A}} \VL  \AR \HL
\NC \m {\MB {V}_{\MB {K}} \SB {N}} \VL Hilbert space \m {\MB {K}^N} over \m {\MB {K}} \VL \m {\MB {K} =\MB {R}} or \m {\MB {C}} \AR \HL
\NC \m {\VNm {\V {a}} _p} \VL \m {\ell_p}-norm of \m {\V {a}} \VL \m {p >0} \AR \HL
\NC \m {\MB {M}_{\MB {K}} \SB {M,N}} \VL collection of \m {M} by \m {N} matrices \VL  \AR \HL
\stoptable
}
% % % % % % % % % % % % % % % % % % % % % % % % % %


\Frame {Channel Model}
{
\I Consider uniform linear array
\Disp {
\NC \V {a} \SB {\psi'}
= \NC \F {1}{\R {N_H}} \sum_{n=0}^{N-1} \Ss {e}^{n \psi' \Ss {i}} \V {u}_n
\in \MB {V}_\MB {C} \SB {N_H} \NR
}

\I Consider the virtual representation of the MIMO channel, where \m {L} is the number of paths, \m {\f_l} the angle of arrival, \m {\th_l} the angle of arrival:
\Disp {
\NC \M{H}
=\NC \sum_{l=0} ^{L-1}
\a_l
\V {a} \SB { \F {d_{\Rm {arr}}} {\l _{\Rm {arr}} \sin \f_l }}
\V {a} \SB { \F {d_{\Rm {arr}}} {\l _{\Rm {arr}} \sin \th_l }}^\Adj
\in \MB {M}_\MB {C} \SB {N_H, N_H} \NR
}
}
% % % % % % % % % % % % % % % % % % % % % % % % % %
\Frame {Precoders and Combiners}
{
\I Tx: Digital precoder \m {\M {F}_B \in \NC \MB {M}_{\MB {C}} \SB {N_R, N_Y}}

\I Tx: Analog precoder \m {\NC \M {F}_R \in \NC \MB {M}_{\MB {C}} \SB {N_H, N_R}}

\I Rx: Digital combiner \m {\M {W}_R \in \NC \MB {M}_{\MB {C}} \SB {N_R, N_H}}

\I Rx: Analog combiner \m {\M {W}_B \in \NC \MB {M}_{\MB {C}} \SB {N_Y, N_R}}

\I With the unity magnitude constraint
\Disp {
\NC \Nm {\M {F}_R \DB {n_H, n_R}}
= \NC 1, \NR
\NC \Nm {\M {W}_R \DB {n_R, n_H}}
= \NC 1, \NR
}

\I With the assumption \m {N_H \gg N_R \gg N_Y}
}
% % % % % % % % % % % % % % % % % % % % % % % % % %
\Frame {Effective Channel}
{
\I Consider zero-mean, unity-variance additive white Gaussian noise (AWGN) \m {\M{Z}}

\I Introduce the effective channel
\Disp{
\NC \M{Y}
:=\M{W}_B \M{W}_R \RB {\M{H} \M{F}_R \M{F}_B +\M{Z}} \NR
}

\I Problem: (a) Recovering \m {\M {H}} with knowledge of \m {\M {H}}?

\I Problem: (b) Designing \m {\M {W}_R}, \m {\M {W}_B}, \m {\M {F}_R}, and \m {\M {F}_B}?
}
% % % % % % % % % % % % % % % % % % % % % % % % % %
\Frame {Working in the Angular Domain}
{
\I Set \m {N_h :=N_H^2,\; N_y :=N_Y^2}

\I Introduce the DFT matrix \m {\M {K}}
\Disp {
\NC \M {K} \in  \NC \MB {M}_{\MB {C}} \SB {N_H, N_H} \NR
\NC \M {K} \DB {n_1, n_2}
= \NC \F {1} {\R {N_H}} \Ss {e}^{2\pi \Ss{i} n_1 n_2 /N_H}, \NR
}

\I Write \m {\M {G} =\NC \M {K}^\Adj \M {H} \M {K}}
}
% % % % % % % % % % % % % % % % % % % % % % % % % %
\Frame {Vectorization}
{
\I Set \m {N_h =N_H^2,\; N_y =N_Y^2}

\I Introduce
\Disp {
\NC \V {y} := \NC \Rm {vec} \SB {\M {Y}} \NR
\NC \M {P} := \NC \RB {\M {F}_B^\Tr \M {F}_R^\Tr \M{K}^\ast} \otimes \RB {\M {W}_B \M {W}_R \M {K}} \NR
\NC \V {g} := \NC \Rm {vec} \SB {\M {G}} \NR
\NC \V {z} := \NC \Rm {vec} \SB {\M {Z}} \NR
}

\I Then we have a vector equation
\Disp {
\NC \V {y}
=\NC \M {P} \V {g} +\V {z} \NR
}
}
% % % % % % % % % % % % % % % % % % % % % % % % % %
\Frame {Proposed Method}
{
\I Input \m{\M {P} \in \MB {M}_{\MB {C}} \SB {N_y, N_h}}, \m{\V {y} \in \MB {V}_{\MB {C}} \SB {N_h}}, \m{\g \geq 0}.

\I Calculate
\Disp {
\NC \Hat {\V {g}}
\LA \NC \startcases
\NC \Min {\V {g}' \in \MB {V}_{\MB {C}} \SB {N_h}}  \MC \VNm {\V {g}'} _1 \NR
\NC \Rm {subject} \; \Rm {to} \Q \MC \VNm {\M {P}^\Adj \RB {\V {y} -\M {P} \V {g}'}} _\infty \leq \g \NR
\stopcases \NR
}

\I Convert \m {\Hat {G} \LA \Rm {vec}^{-1} \SB {\Hat {g}}}

\I Recover \m {\Hat {\M {H}} \LA \M {K} \Hat {\M {G}} \M {K}^\Adj}

\I Output \m {\Hat {\M {H}}}
}
% % % % % % % % % % % % % % % % % % % % % % % % % %


\Frame {Definition: Restricted Isometry Property}
{
\I \m {\V {x}} is called \m {s}-sparse, if there is an index set \m {\MC {A}}, with \m {\# \MC {A} \leq s}, such that
\Disp { 
\NC \V{x} \DB {\MC {A}}
=\NC \V{x} \NR
}

\I For fixed \m {s =0, \dots N_p -1}, we say that \m {\M{\Phi}} satisfies the restricted isometry property (RIP) of sparsity \m {s} with respect to \m {0 \leq \d_s \leq 1}, if, for all \m {s}-sparse \m {\V {x}},
\Disp {
\NC \RB {1-\d_s} \VNm {\V {x}}^2
\leq \NC \VNm {\M{\Phi} \V {x}} _2^2
\leq \RB {1+\d_s} \VNm {\V {x}} _2^2
}

\I That is, \m {\M{\Phi}} is \quotation {almost unitary} up to \quotation {relative error} \m {\d_s}.
}
% % % % % % % % % % % % % % % % % % % % % % % % % %
\Frame {Proof Strategy}
{
\I Show that \m {\V{g}} is almost sparse

\I Show that \m {\M{G}} is almost sparse

\I Find \m {\MB {E} \SB {\VNm {\M{P} \V{u}} _2 ^2}} and \m {\MB {E} \SB {\VNm {\M{P} \V{u}} _2 ^4}}

\I Bound the prob.\ that \m {\Nm {\VNm {\M{P} \V{u}} _2 ^2 -\VNm {\V{u}} _2 ^2} \geq \e \VNm {\V{u}} _2 ^2} by Chebyshev ineq., thus confirming the RIP of \m {\M{P}}

\I Substitute \m {\VNm {\V {g} _{\MC {K}}} _1} into the original Dantzig Selector Proof
}
% % % % % % % % % % % % % % % % % % % % % % % % % %
\Frame {\m {\V{g}} is Almost Sparse}
{
\I Largest \m {S} positions of \m {\V{g} _\MC {A} =\MC {S} \SB {\V{g}, S}}

\I Next largest \m {S} positions of \m {\V{g} _\MC {K} =\MC {C} \SB {\V{g}, S}}

\I Remaining positions of \m {\V{g} _\MC {B} =\MC {S} \SB {\V{g} _\MC {K}, S}}

\I Claim:
\Disp {
\NC R
:= \NC \VNm {\MC {C} \SB {\M {K}^\Adj \V {a} \SB {\f}, s}} _2
\leq \m {\F{1}{3 \R{N_H}}} \NR
}
}
% % % % % % % % % % % % % % % % % % % % % % % % % %
\Frame {Proof (1 of 2)}
{
\Disp {
\NC D \SB {\psi'}
:= \NC \sum_{n_H=0}^{N_H-1} \Ss {e}^{i n_H \psi'},\Q
-\pi \leq \psi' \leq \pi \NR
\NC \RB {\M {K}^\Adj \V {a} \SB {\f}} \DB {n_H}
=\NC \F {1}{N_H} D \SB {\psi \SB {\f, n_H}} \NR
\NC \Nm {D \SB {\psi'}}
= \NC \F {\Nm {\sin \SB {N_H \psi'/2}}}{\Nm {\sin \SB {\psi' /2}}} \NR
\NC \leq \NC B \SB {\psi'} := \F {48}{\Nm {\psi'^2 -24} \Nm {\psi'}} \NR
}

1. Definition

2. Trivial

3. Expansion around \m {\psi' =0}
}
% % % % % % % % % % % % % % % % % % % % % % % % % %
\Frame {Proof (2 of 2)}
{
\Disp {
\NC R^2
\leq \NC \F {1}{N_H^2} \D \F {N_H}{2\pi} \D 2 \int_{\pi s/N_H}^{\pi} B \SB {\psi'} ^2 d \psi' \NR
\NC = \NC \F {2304} {N_H \pi^6}
\int _{s /N_H} ^1 \F{1} {(24/\pi^2 -x'^2)^2 x'^2} dx' \NR
\NC =\NC \F{1} {2\pi^2 N_H}
\RB {
  -\F {8} {u}
+\F {4\pi^2 u} {24 -\pi^2 u^2}
+\R{6} \pi \tanh^{-1} \SB {\F {\pi u} {2\R{6}}}
}
\Bigg \| _{s/N_H} ^1 \NR
}

1. Approximating sum of the largest \m {S} terms by integral

2. Direct calculation
}
% % % % % % % % % % % % % % % % % % % % % % % % % %
\Frame {\m {\M{G}} is Almost Sparse}
{
\I Let \m {\MC {C} \SB {\M {G}, s^2 L}} also denote the largest \m {s^2 L} entries of \m {\M{G}}

\I Claim:
\Disp {
\NC R
:= \NC \VNm {\MC {C} \SB {\M {G}, s^2 L}} _2
\leq \m {\F{1}{3 \R{N_H}}} \NR
}
}
% % % % % % % % % % % % % % % % % % % % % % % % % %
\Frame {Proof}
{
\Disp{
\NC \VNm {\sum _{l=0} ^{L-1} \a_l \V {a} \SB {\f_l} \V {a} \SB {\th_l} ^\Adj } _F
\leq \NC
\sum _{l=0} ^{L-1}
\Nm {\a_l} \VNm {\V {a} \SB {\f_l}} _2
\VNm {\V {a} \SB {\th_l}} _2 \NR
\NC \MB {E} \SB {\VNm {\MC {C} \SB {\sum _{l=0} ^{L-1} \a_l \V {a} \SB {\f_l} \V {a} \SB {\th_l} ^\Adj}} _F}
\leq \NC \F {1} {9 \R{N_H}} \MB {E} \SB {\sum _{l=0} ^{L-1} \Nm {\a_l}}
\leq \F {\R{\pi} L} {3 \R{N_H}} \NR
}

1. Triangle ineq., Cauchy ineq.\

2. Since \m {\a_l} are i.i.d.\ Gaussian
}
% % % % % % % % % % % % % % % % % % % % % % % % % %
\Frame {Design of \m {\M{F}_B}}
{
\I Set each entry of \m {\M{F}_B} to be i.i.d.\ Gaussian r.v.\ with mean 0, standard deviation \m {1/2}, multiplied by \m {\l_B >0}

\I The magnitude \m {\M{F}_B \DB {n_R, n_Y}} follows Rayleigh distribution, having
\Disp {
\NC M_{B,2}
:=\NC \MB{E} \SB {\Nm {\M{F}_B \DB {n_R, n_Y}}^2}
=\l_R^2 \NR
\NC M_{B,4}
:=\NC \MB{E} \SB {\Nm {\M{F}_B \DB {n_R, n_Y}}^4}
=2 \l_R^4 \NR
}

\I We have \m {\MB{E} \SB {\M{F}_B \DB {n_R, n_Y}^2} =0}
}
% % % % % % % % % % % % % % % % % % % % % % % % % %
\Frame {Design of \m {\M{F}_R}}
{
\I Let \m {\M{F}_R \DB {n_H, n_R}} be uniformly distributed on the unit circle on the complex plane, giving
\Disp{
\NC M_{R,2}
:=\NC \MB{E} \SB {\Nm {\M{F}_R \DB {n_H, n_R}}^2}
=\l_R^2 \NR
\NC M_{R,4}
:=\NC \MB{E} \SB {\Nm {\M{F}_R \DB {n_H, n_R}}^4}
=\l_R^4 \NR
}

\I We have \m {\MB{E} \SB {\M{F}_R \DB {n_H, n_R}^2} =0}
}
% % % % % % % % % % % % % % % % % % % % % % % % % %
\Frame {Indicator Function}
{
\I To make expressions more compact, introduce the indication function \m {\i}, which \m {=1} only when the manner that arguments repeat agrees exactly with the subscripts

\I For example \m {\i_{2} \SB {7,7} =1}, and \m {\i_{2,2} \SB {0,3,3,0} =1}, but \m {\i_{2} \SB {5, 6} =0}

\I That is, for some injective \m {\s:\; \CB {0, \ldots, M-1} \mapsto \CB {0, \ldots, N-1}},
\Disp {
\NC \NC \i_{a_0, \ldots, a_{N-1}} \SB {x_0, \ldots, x_{M-1}} \NR
\NC =\NC \startcases
\NC 1, \MC \Q
x_0 \cdots x_{N-1}
=x_{\s \SB {0}} ^{a_0} \cdots x_{\s \SB {M-1}} ^{a_{M-1}} \NR
\NC 0, \NC \Q \Rm {otherwise} \NR
\stopcases \NR
}

}
% % % % % % % % % % % % % % % % % % % % % % % % % %
\Frame {Suffice to Ignore \m {\M{K}}}
{
\I Recall \m {\M {P} := \RB {\M {F}_B^\Tr \M {F}_R^\Tr \M{K}^\ast} \otimes \RB {\M {W}_B \M {W}_R \M {K}}}

\I Define \m {\M{Q} :=\RB {\M {F}_R^\Tr \M {F}_B^\Tr} \otimes \RB {\M {W}_B \M {W}_R}}, then
\Disp {
\NC \V{u} ^\Adj \M{P} ^\Adj \M{P} \V{u}
= \NC \V{u} ^\Adj \M{Q} ^\Adj \M{Q} \V{u} \NR
}

\I Thus
\Disp {
\NC \VNm {\M{P} \V{u}} _2
= \NC \VNm {\M{Q} \V{u}} _2 \NR
}
}
% % % % % % % % % % % % % % % % % % % % % % % % % %
\Frame {Expectation of Some Products}
{
\I Set \m {F :=\NC \M {F}_B \M {F}_R, W :=\NC \M {W}_R \M {W}_B}, then
\Disp {
\NC \NC \MB{E} \SB {\M{F} \DB {n_H, n_Y} \M{F} \DB {n_H', n_Y'}} \NR
\NC = \NC \MB{E} \SB {\M{W}^\Adj \DB {n_Y, n_H} \M{W}^\Adj \DB {n_Y', n_H'}} \NR
\NC = \NC \i_2 \SB {n_H, n_H'} \D \i_2 \SB {n_Y, n_Y'} \D N_R \l_R^2 \l_B^2 \NR
}

\I Similarly
\Disp {
\NC \NC \MB{E}
\SB {
\M{F} \DB {n_H, n_Y}
\M{F} \DB {n_H', n_Y'}
\M{F} \DB {n_H'', n_Y''}
\M{F} \DB {n_H''', n_Y'''}
} \NR
\NC =\NC \MB{E}
\SB {
\M{W} ^\Adj \DB {n_Y, n_H}
\M{W} ^\Adj \DB {n_Y', n_H'}
\M{W} ^\Adj \DB {n_Y'', n_H''}
\M{W} ^\Adj \DB {n_Y''', n_H'''}
} \NR
\NC = \NC
\i_4 \SB {n_H, n_H', n_H'', n_H'''}
\D \i_4 \SB {n_Y, n_Y', n_Y'', n_Y'''}
\D 2 N_R^2 \l_R^4 \l_B^4 \NR
\NC \NC \FourQ +\i_{2,2} \SB {n_H, n_H', n_H'', n_H'''}
\D \i_{2,2} \SB {n_Y, n_Y', n_Y'', n_Y'''}
\D N_R^2 \l_R^4 \l_B^4 \NR
}
}
% % % % % % % % % % % % % % % % % % % % % % % % % %
\Frame {Finding \m {\MB {E} \VNm {\M{P} \V{u}} _2 ^2}}
{
\Disp{
\NC \NC \MB{E} \SB {\VNm {\M{Q} \V{u}} _2 ^2} \NR
\NC = \NC \MB{E} \SB {
  \sum_{n_y=0}^{N_y-1}
  \Nm {\sum_{n_h=0}^{N_h-1} \M{Q} \DB {n_y, n_h} ^\ast \V{u} \DB{n_h}} ^2
} \NR
\NC = \NC
2 \sum_{n_y=0}^{N_y-1}
\sum_{\Stack { n_h, n_h'=0 \NR n_h <n_h' }}^{N_h-1}
\MF {Re} \SB{
   \MB{E} \SB {\M{Q} \DB {n_y, n_h}^\ast}
   \MB{E} \SB {\M{Q} \DB {n_y, n_h'}}
}
\V{u} \DB{n_h} \V{u} \DB{n_h'} ^\ast
\NR
\NC \NC \FourQ +\sum_{n_y=0}^{N_y-1}
   \sum_{n_h=0}^{N_h-1} \MB{E} \SB {\Nm {\M{Q} \DB {n_y, n_h}}^2} \Nm {\V{u} \DB{n_h}}^2
\NR
\NC = \NC N_y \D N_R^2 \l_B^4 \l_R^4  \D \VNm {\V{u}} _2 ^2 \NR
}
}
% % % % % % % % % % % % % % % % % % % % % % % % % %
\Frame {Finding \m {\MB {E} \VNm {\M{P} \V{u}} _2 ^4} (1 of 7)}
{
\Disp{
\NC \MB{E} \SB {\VNm {\M{Q} \V{u}} _2 ^4}
=\NC \MB{E} \SB {
  \RB {
    \sum_{n_R=0}^{N_R-1}
    \Nm {\sum_{n_Y=0}^{N_Y-1} \M{Q} \DB {n_R, n_Y} ^\ast \V{u} \DB{n_Y}} ^2
  } ^2
} \NR
\NC = \NC E_1 +E_2 +E_3 +E_4 +E_5 \NR
}

As follows...
}
% % % % % % % % % % % % % % % % % % % % % % % % % %
\Frame {Finding \m {\MB {E} \VNm {\M{P} \V{u}} _2 ^4} (2 of 7)}
{
\Disp {
\NC E_1
=\NC \sum_{n_y=0}^{N_y-1} \sum_{n_h=0}^{N_h-1}
\Nm {\M{Q} \DB {n_y, n_h}} ^4
\Nm {\V{u} \DB{n_h}} ^4 \NR
\NC =\NC N_Y^2 \D 4 N_Y ^{-4} \D \VNm {\V{u}} _4 ^4 \NR
\NC \leq \NC 4 N_Y ^{-2} \VNm {\V{u}} _2 ^4 \NR
}
}
% % % % % % % % % % % % % % % % % % % % % % % % % %
\Frame {Finding \m {\MB {E} \VNm {\M{P} \V{u}} _2 ^4} (3 of 7)}
{
\Disp {
\NC E_2
= \NC \sum _{\Stack { n_y, n_y' =0 \NR n_y <n_y' }}^{N_y-1}
\sum _{n_h =0}^{N_h-1}
\Nm {\M{Q} \DB {n_y, n_h}} ^2
\Nm {\M{Q} \DB {n_y', n_h}} ^2
\Nm {\V{u} \DB{n_h}} ^4 \NR
\NC = \NC N_Y \D \F{1}{2} N_Y \RB{N_Y-1} \D N_Y ^{-4} \VNm {\V{u}} _2 ^4 \NR
\NC \NC \FourQ + \F{1}{2} N_Y \RB{N_Y-1} \D N_Y^2 \D 4 N_R ^2 N_Y ^{-4} N_R ^{-4} \VNm {\V{u}} _2 ^4 \NR
\NC \leq \NC 4 \VNm {\V{u}} _2 ^4 \NR
}
}
% % % % % % % % % % % % % % % % % % % % % % % % % %
\Frame {Finding \m {\MB {E} \VNm {\M{P} \V{u}} _2 ^4} (4 of 7)}
{
\Disp {
\NC E_3
=\NC \sum _{n_y, n_y' =0}^{N_y-1}
\sum _{\Stack { n_h, n_h' =0 \NR n_h <n_h' }}^{N_h-1}
\Nm {\M{Q} \DB {n_y, n_h}} ^2
\Nm {\M{Q} \DB {n_y', n_h'}} ^2
\Nm {\V{u} \DB{n_h}} ^2
\Nm {\V{u} \DB{n_h'}} ^2 \NR
\NC \leq \NC N_Y^2 \D \F{1}{2} N_H \RB{N_H-1} \D N_Y ^{-4} \D \RB {\VNm {\V{u}} _2 ^4 -\VNm {\V{u}} _4 ^4} \NR
\NC \leq \NC 4 N_Y ^{-1} \VNm {\V{u}} _2 ^4 \NR
}
}
% % % % % % % % % % % % % % % % % % % % % % % % % %
\Frame {Finding \m {\MB {E} \VNm {\M{P} \V{u}} _2 ^4} (5 of 7)}
{
\Disp {
\NC E_4
=\NC \sum _{n_y =0}^{N_y-1}
\sum _{\Stack { n_h, n_h' =0 \NR n_h <n_h' }}^{N_h-1}
\Nm {\M{Q} \DB {n_y, n_h}} ^2
\Nm {\M{Q} \DB {n_y, n_h'}} ^2
\Nm {\V{u} \DB{n_h}} ^2
\Nm {\V{u} \DB{n_h'}} ^2 \NR
\NC \leq \NC N_Y \D \F{1}{2} N_H \RB{N_H-1} \D N_Y ^{-4} \D \RB {\VNm {\V{u}} _2 ^4 -\VNm {\V{u}} _4 ^4} \NR
\NC \leq \NC 4 N_H ^2 N_Y ^{-3} \VNm {\V{u}} _2 ^4 \NR
}
}
% % % % % % % % % % % % % % % % % % % % % % % % % %
\Frame {Finding \m {\MB {E} \VNm {\M{P} \V{u}} _2 ^4} (6 of 7)}
{
\Disp {
\NC E_5
= \NC \sum _{
   \Stack {
     0 =n_h <n_h' \NR
     0 =m_h <m_h' \NR
     n_h +m_h' =n_h' +m_h \NR
     \Fl {n_h/N_H} =\Fl {n_h'/N_H} \NR
     \Fl {m_h/N_H} =\Fl {m_h'/N_H}
   }
} ^{N_H-1}
\sum _{
  \Stack {
     0 =n_y <n_y' \NR
     \Fl {n_y/N_Y} =\Fl {n_y'/N_Y}
  }
} ^{N_y-1} \NR
  \NC \NC \FourQ
  \M{Q} \DB {n_y, n_h}
  \M{Q} \DB {n_y, n_h'}
  \M{Q} \DB {n_y', m_h}
  \M{Q} \DB {n_y', m_h'} \NR
  \NC \NC \FourQ
  \V{u} \DB{n_h}
  \V{u} \DB{n_h'}
  \V{u} \DB{m_h}
  \V{u} \DB{m_h'} \NR
\NC = \NC \dots \NR
}
}
% % % % % % % % % % % % % % % % % % % % % % % % % %
\Frame {Finding \m {\MB {E} \VNm {\M{P} \V{u}} _2 ^4} (7 of 7)}
{
\Disp {
\NC =\NC N_Y \D \F{1}{2} N_H \RB{N_H-1} \D \F{1}{2} N_Y \RB{N_Y-1} \D N_H^2 \D N_Y ^{-4} \NR
\NC \NC \FourQ
\D N_H^{-2} \sum_{n_h, n_h' =0} ^{N_h-1} \sum_{m_h, m_h' =0} ^{N_h-1} \NR
\NC \NC \i_{1,1,1,1} \SB {n_h, n_h', m_h, m_h'}
\V{u} \DB{n_h} \V{u} \DB{n_h'} \V{u} \DB{m_h} \V{u} \DB{m_h'} \NR
\NC \leq \NC \F{1}{6} N_Y ^{-1} \VNm {\V{u}} _2 ^4 \NR
}
}
% % % % % % % % % % % % % % % % % % % % % % % % % %
\Frame {Confirming the RIP of \m {\M{P}}}
{
\I Finding the variance
\Disp{
\NC \Ss{Var} \SB {\VNm {\M{P} \V{u}} _2 ^2}
\leq \NC \RB {4 N_H^2 N_Y^{-3} -1} \VNm {\V{u}} _2 ^4 \NR
}

\I Using the Chebyshev Bound
\Disp{
\NC \MB{P}
\SB {
  \Nm {\VNm {\M{P} \V{u}} _2 ^2 -\VNm {\V{u}} _2 ^2}
  \geq \e \VNm {\V{u}} _2 ^2
}
\leq \NC \F {1} {\e^2} \RB {4 N_H^2 N_Y^{-3} -1} \NR
}
}


% % % % % % % % % % % % % % % % % % % % % % % % % %
\Frame {Expected Error of DS}
{
\I Let \m {\V {y}}, \m {\M {P}}, \m {\V {g}}, \m {\hat {\V {g}}}, \m {\V {d}} be defined as above.
Then, with \m {S =s^2 L},
\Disp {
\NC \VNm {\V {d}} _2
\leq \NC 32 s^2 L \log N_H \NR
}

\I Moreover, under the design condition
\Disp {
\NC \RB {2 N_H}^2
\approx \NC N_Y^3 \NR
}

\I The bound holds for probability \m {p}, with
\Disp {
\NC 1 -p
\leq \NC 16 \D {12}^S \d_{S}^{-S-2} \RB {N_H^2 N_Y^{-3} -\F{1}{4}} \NR
}

\I \m {\Rightarrow} Is \m {16 \D {12}^S \d_{S}^{-S-2}} unreasonably large?

\I \m {\Rightarrow} Can \m {N_H^2 N_Y^{-3} -1/4} be made small?
}
% % % % % % % % % % % % % % % % % % % % % % % % % %
\Frame {Technical Lemmata}
{
{\tfx
\I Define \m {\V {d} := \Hat {\V {g}} -\V {g}}

\I From optimality of \m {\Hat {\V {g}}}: \m {
\NC \VNm {\V {d} _{\MC {K}}} _1
\leq \NC \VNm {\V {d} _{\MC {A}}} _1 +2\VNm {\V {g} _{\MC {K}}} _1 \NR
}

\I From central limit theorem: \m {
\NC \MB {E} \SB {\Nm {\IP {\V {z}, \M {P} \DB {:, n_h}}}}
\leq \NC  2 \R {\log N_h} \NR
}

\I From above: \m {
\NC \VNm {\M {P}^\Adj \M {P} \V {d}} _\infty
\leq \NC  4 \R {\log N_h} \NR
}

\I From DS paper: \m {
\NC \VNm {\V {d} _{\MC {AB}}} _2
\leq \NC \F {1}{1-\d_{2S}} \VNm {P _{\MC {A} \MC {B}}^\Tr P d} _2 +\F {\d_{3S}}{\RB {1-\d_{2S}} \R {S}} \VNm {d_{\MC {K}}} _1 \NR
}

\I From DS paper: \m {
\NC \VNm {\V {d}} _2^2
\leq \NC \VNm {\V {d} _{\MC {A} \MC {B}}} _2^2 +\F {1}{S} \VNm {\V {d} _{\MC {K}}} _1^2 \NR
}
}
}
% % % % % % % % % % % % % % % % % % % % % % % % % %
\Frame {Proof (1 of 2)}
{
\I By \m {\ell_p}-norm ineq., by Lemma for \m {\VNm {\M {P}^\Tr \M {P} \V {d}} _\infty},
\Disp {
\NC \VNm {\M {P} _{\MC {A} \MC {B}}^\Tr \M {P} \V {d}} _2
\leq \NC \R {S} \VNm {\M {P}^\Tr \M {P} \V {d}} _\infty
\leq 4 \R {S \log N_h} \NR
}

\I By \m {\ell_p}-norm ineq., by Lemma for \m {\VNm {\V {d} _{\MC {AB}}} _2}, by above,
\Disp {
\NC \VNm {\V {d}_{\MC {A}}} _1
\leq \NC \R {S} \VNm {\V {d}_{\MC {A}}} _2 \NR
\NC \leq \NC \d_{3S} \RB {1+\d_{2S}} \VNm {\V {d} _{\MC {K}}} _1
+4 \RB {1+\d_{2S}} S \R {\log N_h} \NR
}

\I By Lemma for \m {\VNm {\V {d} _{\MC {K}}} _1}, by above, by dropping terms,
\Disp {
\NC \VNm {\V {d} _{\MC {K}}} _1
\leq \NC 2 \RB {1+\d_{3S}} \VNm {\V {g} _{\MC {K}}} _1
+4 \RB {1+2\d_{3S}} S \R {\log N_h} \NR
}
}
% % % % % % % % % % % % % % % % % % % % % % % % % %
\Frame {Proof (2 of 2)}
{
\Disp {
\NC \VNm {\V {d}} _2^2
\leq \NC 16 S \log N_h
+8 \d_{3S} \R {\log N_h} 
+\F {\d_{3S}^2} {S} \VNm {\V {d} _{\MC {K}}} _1 \NR
\NC \leq \NC 32 s^2 L \log N_H
+\F {8 \R {2\pi}} {3} \d_{3S} L \R {\F {\log N_H} {N_H}}
+\F {\pi} {9} \F {L} {s^2 N_H} \NR
\NC \leq \NC 32 s^2 L \log N_H \NR
}

1. Plug in the bound for \m {\VNm {\V {d} _{\MC {K}}} _1}

2. Plug in \m {S =s^2 L} and \m {N_h =N_H^2}
}



\input {./4_slide.tex}
\Frame {Conclusion}
{
\I We are concerned with effective and estimation of MIMO mm-wave channel by exploiting its sparsity

\I Towards that end, We adopt The Dantzig Selector (DS) rather than the popular Orthogonal Matching Pursuit (OMP)

\I We generalized DS for complex vectors, and carried it out in the angular domain

\I Analogous to original DS proof, we bound the expected square error, and the successful probability is also bounded by a explicit constant
}

\page [yes] % % % % % % % % % % % % % % % % % % % % % % % % % %

\Frame {Future Work}
{
\I Check the proof carefullly, especially approximation

\I Finish the simulation program

\I Compare the probability bound with simulation result

\I Also implement OMP and compare they two
}

\page [yes] % % % % % % % % % % % % % % % % % % % % % % % % % %

\Frame {References}
{
{\tfxx
\I D Achlioptas (2001). Database-friendly random projections. In Proceedings of the twentieth ACM SIGMOD-SIGACT-SIGART symposium on Principles of database systems (pp. 274-281). ACM.

\I W U Bajwa, J Haupt, G Raz, and R Nowak (2008), ``Compressed Channel Sensing'', 2008 42nd Annual Conference on Information Sciences and Systems.

\I R Baraniuk, M Davenport, R DeVore, and M Wakin (2008), ``A Simple Proof of the Restricted Isometry Property for Random Matrices'', \It{Constructive Approximation} \Bf{28}: 253–263

\I E J Cand\`es and T Tao (2005), ``Decoding by Linear Programming'', \It{IEEE Transactions on Information Theory}, Vol.51, No.12.

\I E Cand\`es and T Tao (2007), ``The Dantzig Selector: Statistical Estimation when \m {p} is Much Larger than \m {n}'', \It{The Annals of Statistics}, Vol.35, No.6.
}
}



\stoptext

\page [yes] % % % % % % % % % % % % % % % % % % % % % % % % % %


\Frametitle {Program: The Dantzig Selector}
Let \m {} and \m {} be given, and \m {} be fixed.
Find \m {} with
\Disp{
\hat{\V{h}} =\min_{\V{h}'} \quad &\|\V{h}'\|_1 \\ \NT
\RM{subject}\; \RM{to} \quad &\|\M{P}^\H (\M{P} \V{h'} -\V{y})\|_\infty \leq \gamma
}



\page [yes] % % % % % % % % % % % % % % % % % % % % % % % % % %


\Frametitle {Performance Guarantee of DS}


\I Now, consider a linear transformation with noise corruption,
\Disp{
\V{y} =\M{P} \V{h} + \V{z}
}
where \m {} is i.i.d.\ standard Gaussian.

\I They showed that the mean square error \m {} is bounded with overwhelming probability.

\I Furthermore, this \m {}-minimization problem with \m {}-constraint may be recast as a linear program (LP), lending convex programming technique applicable.





\page [yes] % % % % % % % % % % % % % % % % % % % % % % % % % %


\Frametitle {Definition: Restricted Isometry Property}


\I Consider \m {}, with unity-\m {}-norm columns.
For \m {}, we say that \m {} satisfies the restricted isometry property (RIP) of sparsity \m {} with respect to \m {}, if, for all \m {}-sparse \m {}, for all \m {} with \m {},
%
\Disp{
(1-\d_s) \|\V{x}\|^2
\leq \|P_{\MC{T}} \V{x}\|_2^2
\leq (1+\d_s) \|\V{x}\|_2^2
}
\I RIP is essentially saying that \m {} is ``almost unitary'' up to ``relative error'' \m {}.





\page [yes] % % % % % % % % % % % % % % % % % % % % % % % % % %


\Frametitle {Orthogonal Matching Pursuit}


\I Afterwards, scholars (Tropp and Gilbert 2007b) proposed a greedy algorithm called Orthogonal Matching Pursuit (OMP)
\I Here, we pick up the columns of the sensing matrix \m {} greedily, hoping to correspond to the support of \m {}, thus recovering the original signal.
\I An i.i.d.\ random sensing matrix may perform sufficiently well, and may even recover the original signal in an overwhemingly probability too.





\page [yes] % % % % % % % % % % % % % % % % % % % % % % % % % %


\Frametitle {Recent Literature on Compressive Channel Sensing}


\I Scholars has since favored OMP rather than DS, let alone other sparse algorithm, without clear justification
\I Previous work simply assumes the norm of the channel matrix is bounded in some way, and dependency on the sparsity of the channel parameters is unknown
\I OMP's requirement on the sensing matrix (elementwise i.i.d.\ Gaussian) seems to be more restrictive than DS's (RIP)
\I The quantization of angle in generating may be a problem, and that is difficult to analyze in OMP's setting





\page [yes] % % % % % % % % % % % % % % % % % % % % % % % % % %


\Frametitle {Our Work}


\I We will use a modified DS rather than OMP, done on the beamspace rather than the spatial domain, and involving complex numbers rather than real numbers.
\I We will show an explicit bound, where the sparsity of the virtual channel matrix is depends explicitly on the number of paths of the channel.
\I Bounding the beamspace sparsity, accordingly the concern of quantization error of the virtual channel's phase angle has been incorporated in our proof of the bound.
\I We will derive explicitly the SOCP problem and afterwards its dual problem, and wrote a proof-of-concept but efficient code.





\page [yes] % % % % % % % % % % % % % % % % % % % % % % % % % %


\Frametitle {Channel Model}


\I The response of uniform linear array is
\Disp{
\V{a} (\psi')
=\F{1}{\R{N_H}} \sum_{n=1}^{N} \RM{e}^{n \psi' i} \V{u}_n
\in \MB{V}_\MB{C} (N_H)
}
\I Let the virtual angle of departure and arrival be defined as thus to simplify expression
\Disp{
\f_l =\dfrac{d_{\RM{arr}}} {\lambda} \sin \f_l', \quad \th_l =\dfrac{d_{\RM{arr}}} {\lambda} \sin \th_l'
}
\I Let there be \m {} paths.
The channel matrix is, then,
\Disp{
H
=\sum_{l=0}^L \alpha_l \V{a} (\f_l) \V{a} (\th_l)^\H
\in \MB{M}_\MB{C} (N_H, N_H)
}





\page [yes] % % % % % % % % % % % % % % % % % % % % % % % % % %


\Frametitle {Precoder Setting}


\I We consider the hybrid configuration at both transmitter and receiver end.
In the transmitter end, there are digital precoder \m {} and analog precoder \m {}.
In the receiver end, there are digital combiner \m {} and analog combiner \m {}.
\Disp{
\M{F}_B \in &\MB{M}_{\MB{C}} (N_R, N_Y) \\
\M{F}_R \in &\MB{M}_{\MB{C}} (N_H, N_R) \\
\M{W}_R \in &\MB{M}_{\MB{C}} (N_R, N_H) \\
\M{W}_B \in &\MB{M}_{\MB{C}} (N_Y, N_R)
}
\I Recall the constraint of magnitude for analog precoders:
\Disp{
|\M{F}_R (n_h, n_r)| =1, \quad |\M{W}_R (n_r, n_h)| =1, \\
n_h =1, \dotsc N_H, \quad n_r =1, \dotsc N_R \NT
}
\I We also assume
\Disp{
N_Y \ll N_R \ll N_H
}





\page [yes] % % % % % % % % % % % % % % % % % % % % % % % % % %


\Frametitle {Effective Channel}


\I We have the effective channel \m {}
\I We may estimate \m {} with \m {}, \m {}, using pilog signal as unit vectors \m {}.
\I It remains to recover \m {} with knowledge of \m {}, while \m {}, \m {}, \m {}, and \m {} are in our control






\page [yes] % % % % % % % % % % % % % % % % % % % % % % % % % %


\Frametitle {Vectorization}
Previous literature usually approaches the problem as
\Disp{
\V{h}
:= &\RM{vec} (\M{H})
\in \MB{V}_{\MB{C}} (N_g) \\
\V{y}
:= &\RM{vec} (\M{Y})
\in \MB{V}_{\MB{C}} (N_y) \\
\V{z}^\star
:= &\RM{vec} (\M{Z})
\in \MB{V}_{\MB{C}} (N_y) \\
\M{P}^\star
:= &(\M{F}_R^\Tr \M{F}_B^\Tr) \otimes (\M{W}_B \M{W}_R)
\in \MB{M}_{\MB{C}} (N_y, N_g).
}
and for short we set \m {} and \m {},
so that
\Disp{
\V{y} =\M{P}^\star \V{h} +\V{z}^\star
}



\page [yes] % % % % % % % % % % % % % % % % % % % % % % % % % %


\Frametitle {Beamspace Channel Representation}
Let \m {} be the DFT matrix.
If we write \m {}, i.e.\ the beamspace (i.e., spatial frequency domain) representation of \m {}, then
\Disp{
\M{Y}
:=&\M{W}_B \M{W}_R \M{K} ( \M{G} \M{K}^\H \M{F}_R \M{F}_B +\M{K}^\H \M{Z} )
\in \MB{M}_{\MB{C}} (N_Y, N_Y) \\
\M{P}
:=&(\M{F}_B^\Tr \M{F}_R^\Tr \M{K}^\ast) \otimes (\M{W}_B \M{W}_R \M{K})
\in \MB{M}_{\MB{C}} (N_y, N_g)
}
where accordingly
\Disp{
\V{g} := &\RM{vec} (\M{G}) \in \MB{M}_{\MB{C}} (N_g) \\
\V{z} := &\RM{vec} (\M{K}^\H \M{Z}) \in \MB{V}_{\MB{C}} (N_y)
}
so that
\Disp{
\V{y}
=\M{P} \V{g} +\V{z}
}



\page [yes] % % % % % % % % % % % % % % % % % % % % % % % % % %


\Frametitle {Proposed Program}


\I Let \m {} be given.
Then, with parameter \m {} specified, find
%
\Disp{
\hat{\V{g}}
=\min_{\V{g}'} &\|\V{g}'\|_1 \quad \\
\RM{subject}\; \RM{to}\quad
&\|\M{P}^\H (\V{y} -\M{P} \V{g}')\|_\infty \leq \gamma
}
\I And convert \m {} back to matrix form as
\Disp{
\hat{G} =\RM{vec}^{-1} (\hat{g})
}
\I And finally recover the estimated \m {} in the space domain as
\Disp{
\hat{\M{H}} =\M{K} \hat{\M{G}} \M{K}^\H.
}





\page [yes] % % % % % % % % % % % % % % % % % % % % % % % % % %


\Frametitle {Lemma: \m {} is Almost Sparse}
For arbitrary \m {}, for \m {}, \m {} is almost-\m {}-sparse with \m {}-residue \m {} to be
\Disp{
R
   \leq \F{1}{\pi} \log \F{N_H}{L}.
}



\page [yes] % % % % % % % % % % % % % % % % % % % % % % % % % %


\Frametitle {Proof (1/3)}
\Disp{
D (\f')
:=&\left| \sum_{n=0}^{N_H-1} \RM{e}^{i n \f'} \right|
=\F{|\sin (N_H \f'/2)|}{|\sin (\f' /2)|} \\
\left| x -\F{x^3}{6} \right|
\leq &\sin x, \quad -\pi \leq x \leq \pi \\
\left| D (\f') \right|
= &\F{48}{|\f'^2 -24| |\f'|}
}


\I Definition and arrangement
\I Can be shown with basic calculus
\I Plug the previous eqn.\ into \m {}





\page [yes] % % % % % % % % % % % % % % % % % % % % % % % % % %


\Frametitle {Proof (2/3)}
\Disp{
\M{K}^\H \V{a}(\f) (k)
=&\F{1}{N_H} D \left( \f -\F{2 \pi k} {N_H} \right) \\
R(\eta)
:=&\F{1}{N_H} \sum_{n_H =s}^{N_H -1} D \left( \eta +\F{2 \pi n_H} {N_H} \right) \\
R(\eta) -\F{2\pi} {N_H}
\leq &\F{1}{N_H} N_H \int_{\f'=2\pi L/N_H}^{2\pi} |D(\f')| d \f'
}


\I Straightforward by definition
\I Definition
\I Approximation of rectangle to integral





\page [yes] % % % % % % % % % % % % % % % % % % % % % % % % % %


\Frametitle {Proof (3/3)}
\Disp{
R(\f)
\leq &\F{1}{2\pi} \int_{\f'=2\pi L/N_H}^{2\pi} \F{48}{(24 -\f'^2) \f'} d \f'
+\F{2\pi} {N_H} \NT \\
=&\F{1}{\pi} \log \F{2\pi N_H}{L}
-\F{1}{N_H} \log \F{4\pi^2 -24} {L^2/N_H^2 -24}
+\F{2\pi} {N_H} \\
\leq &\F{1}{\pi} \log \F{N_H}{L}.
}


\I Plug in above bound for \m {}
\I Calculation
\I Drop the middle term (\m {}) and last term (small)





\page [yes] % % % % % % % % % % % % % % % % % % % % % % % % % %


\Frametitle {Lemma (Tentative): \m {} is Almost Sparse}
Let \m {} and \m {} be uniformly, independently distributed in \m {}.
Then \m {} is almost-\m {}-sparse with \m {}-residue \m {} to be
\Disp{
R
\leq \F{L}{\pi^2} \left( \log \F{N_H}{L} \right)^2
}

\page [yes] % % % % % % % % % % % % % % % % % % % % % % % % % %



\page [yes] % % % % % % % % % % % % % % % % % % % % % % % % % %


\Frametitle {The Main Bound}
Let \m {} be defined as above, then, for overwhelming probability
\Disp{
\|\V{d}\|_2
\leq \F{4}{\pi^4} \d_{3L}^4 (1-2\d_{2L}) L^4 ( \log N_H )^4
}



\stoptext
