
\blank [big, force]

\Title {Dantzig Selector \\ Applied on mm-Wave MIMO: \\ Part I. Expected Error Analysis}
\blank [big]

\Subtitle {Presenter: Tzu-Yu Jeng \\ Advisor: Prof.\ H.J.\ Su}
\blank [big]

\Subsubtitle{May 30, 2019}
\Subsubtitle{Graduate Instute of Commnication Engineering, \\ National Taiwan University}

\page [yes]
% % % % % % % % % % % % % % % % % % % % % % % % % %
\Frame {Organization}
{
\I Problem Configuration

\I Proposed Method

\I Sparsity of Angular Channel Matrix

\I Concentration Inequality and Restricted Isometry of \m {\M{P}}

\I Expected Error of Dantzig Selector

\I Conclusion and Future Work

\I References
}
% % % % % % % % % % % % % % % % % % % % % % % % % %
\Frame {Background}
{
\I Multiple-input-multiple-output (MIMO) wireless communication is expected to be the next-generation r\'egime

\I But channel response has to be known at the receiver to facilitate applications like beamforming and channel calibration

\I And in the MIMO system, estimating wireless channels calls for high complexity

\I Meanwhile, in the millimeter wave r\`egime, which is often used together with MIMO, channel often exhibits sparse properties

\I If the sparsity is exploited, can few observations suffice to estimate the channel?
}
% % % % % % % % % % % % % % % % % % % % % % % % % %
\Frame {Compressive Sensing}
{
\I Such case of channel estimation is facilitated by advances of so-called compressed sensing (CS)

\I CS addresses the situation that the number of model parameters \m {N_p \gg} the number of measurements \m {N_m}.

\I With insufficient (possibly noisy) measurements, under what circumstances can we recover all \m {N_p} parameters?

\I Much work on CS reveals that few observations of the signal may be sufficient for us to reconstruct the signal when it is sparse.

\I In the advent, Cand\`es and Tao (2006) propose the Dantzig Selector (DS) as a possible answer.
}
% % % % % % % % % % % % % % % % % % % % % % % % % %
\Frametitle {Notation}
{
{\tfx
\starttable[|l|l|l|] \HL
\NC Expression \VL Meaning or Definition \VL More Explanation \SR \HL
\NC \m {f \SB {x}} \VL square brackets \VL only for arguments \AR \HL
\NC \m {\V {a}} \VL vector \VL  \AR \HL
\NC \m {\M {A}} \VL matrix \VL  \AR \HL
\NC \m {\V {a} \DB {n}} \VL \m {n}-th component of \m {\V {a}} \VL  \AR \HL
\NC \m {\M {A} \DB {m,n}} \VL \m {m,n}-th entry of \m {\M {A}} \VL  \AR \HL
\NC \m {\MB {V}_{\MB {K}} \SB {N}} \VL Hilbert space \m {\MB {K}^N} over \m {\MB {K}} \VL \m {\MB {K} =\MB {R}} or \m {\MB {C}} \AR \HL
\NC \m {\VNm {\V {a}} _p} \VL \m {\ell_p}-norm of \m {\V {a}} \VL \m {p >0} \AR \HL
\NC \m {\MB {M}_{\MB {K}} \SB {M,N}} \VL collection of \m {M} by \m {N} matrices \VL \m {\MB {K} =\MB {R}} or \m {\MB {C}} \AR \HL
\stoptable
}
}
% % % % % % % % % % % % % % % % % % % % % % % % % %

