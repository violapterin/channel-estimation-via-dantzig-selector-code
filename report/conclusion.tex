
\startchapter [title={Conclusion}]

This article addresses the problem of effective and estimation of the millimeter wave channel by exploiting its sparsity.
We motivated the problem, and provided a rather broad survey of recent literature.
This article adopts The Dantzig Selector (DS) rather than Orthogonal Matching Pursuit (OMP), like many of past literature, as the estimation algorithm.
Rather than exploiting the sparsity of space domain representation as existing literature does, we proceed the estimation algorithm in the spatial frequency domain.
The technique is surely inspiring to future work, as many bounds are ready to be obtained once we change the physical model of the channel.
In addition, we have, as a secondary product, derived the generalization of DS for complex vectors.

Moreover, assuming the virtual channel model according to uniform linear array response, we managed to prove explicitly, following the spirit of Cand\`es and Tao (2007), that the probability that the expected square error is bounded is overwhelming.
We thus have derived a probability bound of performance on a fairly general setting, where the bounding constant is expressed as an explicit function of physical parameters, such as the number of path.

Afterwards we transformed DS into an SOCP, which is quite different as in the original DS (where we have an LP instead).
Simulation is done, and DS's performance is compared to OMP.
{ \color{red} (To be done) }

We cannot think OMP, or indeed every algorithm, as a universal solution.
OMP trades off precision for time and space complexity, and for DS the other way around is true, hence we cannot say which is definitely better than the other.
All in all, different problems requires different technique and algorihm within the current constraint, as is probably true in every discipline of engineering.

\stopchapter


