\startsection [title={Confirming Restricted Isometry of Beamformer}]
\startsubsection [title={Design of Digital Beamformer Entries}]

Our plan is setting the four precoder matrices to be i.i.d.\ random matrix, hoping that the resulting \m {P} has RIP.
Indeed, \m {\d_s \SB {\M{F}_B} \SB {\o_t}} may be found, along similar lines with Achlioptas (2001) and Baraniuk et.\ al.\ (2008).
We prove a concentration inequality, as explained below, and invoke Chebyshev inequality.
But before that, we have to investigate the moments of each entry of \m {\M{F}_B \SB {\o_t}} (resp.\ \m {\M{W}_B \SB {\o_t}}) and that of \m {\M{F}_R \SB {\o_t}} (resp.\ \m {\M{W}_R \SB {\o_t}}).

Let us make it simple and set each entry of \m {\M{F}_B} to be i.i.d.\ Normal r.v.\ with mean 0, standard deviation \m {1/2}, multiplied by a normalizing constant \m {\l_B >0}.
They are (with a slight abuse of \m {\o_t})
\Disp{
\NC d \o_t
=\NC \F {1} {\R {\pi} \l_B} \exp
  \RB {-\l_B^{-2} \MF{Re} \SB {\M{F}_B \DB {n_R, n_Y}} ^2}
d \MF{Re} \SB {\M{F}_B \DB {n_R, n_Y}} \NR[+]
\NC d \o_t
=\NC \F {1} {\R {\pi} \l_B} \exp
  \RB {-\l_B^{-2} \MF{Im} \SB {\M{F}_B \DB {n_R, n_Y}} ^2}
d \MF{Im} \SB {\M{F}_B \DB {n_R, n_Y}} \NR[+]
\NC n_R
= \NC 0, 1, 2, \ldots, N_R -1 \NR
\NC n_Y
= \NC 0, 1, 2, \ldots, N_Y -1 \NR
}

We know that the magnitude \m {\M{F}_B \DB {n_R, n_Y}} follows Rayleigh distribution.
Set for shorthand
\Disp {
\NC M_{B,k}
:=\NC \MB{E} \SB {\Nm {\M{F}_B \DB {n_R, n_Y} \SB {\o_t}}^k} \NR
\NC k =\NC 1, 2, 3, \dots \NR
}

It is known that
\Result
{Lemma}
{
\Disp {
\NC M_{B,k}
=\NC \l_B^k \Gamma \SB {\F{k}{2} +1} \NR[+]
}
}

In particular,
\Disp {
\NC M_{B,2}
=\NC \l_B^2 \NR[+]
\NC M_{B,4}
=\NC 2 \l_B^4 \NR[+]
}

\m {\M{W}_B \DB {n_Y, n_R}} is set in an entirely analolous manner.

\stopsubsection

\startsubsection [title={Design of Analog Beamformer Entries}]

Let \m {\M{F}_R \DB {n_H, n_R}} be uniformly distributed on the unit circle on the complex plane, which gives probability density
\Disp{
\NC d \o_t
= \NC \F {1} {\pi \l_R} \RB {1 -\l_R^{-2} \MF{Re} \SB {\M{F}_R \DB {n_H, n_R}} ^2}^{-1/2}
d \MF{Re} \SB {\M{F}_R \DB {n_H, n_R}} \NR[+]
\NC d \o_t
= \NC \F {1} {\pi \l_R} \RB {1 -\l_R^{-2} \MF{Im} \SB {\M{F}_R \DB {n_H, n_R}} ^2}^{-1/2}
d \MF{Im} \SB {\M{F}_R \DB {n_H, n_R}} \NR[+]
}
Set
\Disp {
\NC M_{R,k}
:=\NC \MB{E} \SB {\Nm {\M{F}_R \DB {n_H, n_R} \SB {\o_t}}^k} \NR
\NC k =\NC 1, 2, 3, \dots \NR
}
Then it is trivial that
\Disp {
\NC M_{R,k}
=\NC \l_R^k \NR[+]
}
In fact, it seems to be possible, from analysis as follows, to require that \m {\M{F}_R \DB {n_H, n_R}} be uniformly distributed only on the discrete set of \m {p}-th roots of unity if \m {p} is big enough.
The point may be exploited in practical application when it is decided to avoid the overhead of generation of random phase.
But we choose such continuous distribution to keep the analysis simple.

\m {\M{W}_B \DB {n_Y, n_R}} is set in an entirely analolous manner.

\stopsubsection

\startsubsection [title={Suffice It to Ignore DFT Matrix}]

\Result
{Definition}
{
Let \m {P} be fixed.
For \m {\MC {T}, \MC {T}' \subset \MC {T} \RB {N_p}}, define the \m {s, s'}-restricted orthogonality constant \m {\tau_{s,s'} \SB {P} >0} to be the smallest number such that
\Disp {
\NC \Nm {\IP {P _{\MC {T}} h, P _{\MC {T}'} h'}}
\leq \NC \tau_{s, s'} \SB {P} \cdot \VNm {h} _2 \VNm {h'} _2 \NR
}
}

From \quotation {Decoding from Linear Programming} (Cand\`es and Tao 2005), Lemma 1.1:

\Result
{Lemma}
{
Let \m {\M{P}} be fixed.
Then \m {\tau_{s, s'} \SB {\M{P}}} is bounded in both direction as follows,
\Disp {
\NC \d_{s+s'} \SB {\M{P}} -\max \SB {\CB {\d_s \SB {\M{P}}, \d_{s'} \SB {\M{P}}}}
\leq \NC \tau_{s, s'} \SB {\M{P}} \NR
\NC \leq \NC \d_{s+s'} \SB {\M{P}} \NR[+]
}
}

Thus \m {\d_s \SB {\M{P}}}, which defines how much the deformation of norm is, also tells us how much the inner product is deformed.
We may well keep track of \m {\d_s \SB {\M{P}}} only.
When \m {\M{P}} is clear, we may suppress it.

But each entry of \m {P} is a linear combination of products, and such reasoning does not work.
If only \m {P} were also an i.i.d.\ random matrix, the resulting proof would be easier.
Is the approach all but lost?

Observe
\Disp {
\NC \M{P} ^\Adj \M{P}
=\NC \RB {
   \RB {\M{K} ^\Tr \M {F}_R^\ast \M {F}_B^\ast}
   \otimes \RB {\M {K} \M {W}_R \M {W}_B}
}
\RB {
   \RB {\M {F}_B^\Tr \M {F}_R^\Tr \M{K}^\ast}
   \otimes \RB {\M {W}_B \M {W}_R \M {K}}
} \NR
\NC =\NC
\RB {\M {F}_B^\Tr \M {F}_R^\Tr \M{K}^\ast \M{K}^\Tr \M {F}_R^\ast \M {F}_B^\ast}
\otimes \RB {\M {W}_B \M {W}_R \M{K} \M{K}^\Adj \M {W}_R^\Adj \M {W}_B^\Adj} \NR
\NC =\NC
\RB {\M {F}_B^\Tr \M {F}_R^\Tr \M {F}_R^\ast \M {F}_B^\ast}
\otimes \RB {\M {W}_B \M {W}_R \M {W}_R^\Adj \M {W}_B^\Adj} \NR
\NC =\NC \RB {
   \RB {\M {F}_R^\ast \M {F}_B^\ast}
   \otimes \RB {\M {W}_R \M {W}_B}
}
\RB {
   \RB {\M {F}_B^\Tr \M {F}_R^\Tr}
   \otimes \RB {\M {W}_B \M {W}_R}
} \NR
\NC =\NC \M{Q} ^\Adj \M{Q} \NR
}
where as we defined before \m {\M{Q} :=\RB {\M {F}_B^\Tr \M {F}_R^\Tr} \otimes \RB {\M {W}_B \M {W}_R}}.

This implies
\Disp {
\NC \V{u} ^\Adj \M{P} ^\Adj \M{P} \V{u}
= \NC \V{u} ^\Adj \M{Q} ^\Adj \M{Q} \V{u} \NR[+]
}
or
\Result
{Lemma}
{
For any instance of \m {\M{P} \SB {\o_t, \o_r}} and \m {\M{Q} \SB {\o_t, \o_r}},
\Disp{
\NC \VNm {\M{P} \V{u}} _2
= \NC \VNm {\M{Q} \V{u}} _2 \NR[+]
}
}
\stopsubsection

\startsubsection [title={Expectation of Products of Matrix Entry}]

To make expressions more compact, introduce the indication function \m {\i} so that it equals 1 only if all (positive integer) argument it receives are unequal when they are seperated by semicolons between them, and equal when they are not.
For example, \m {\i \SB {7, 7, 7, 7} =1}, and \m {\i \SB {0, 0; 3, 3, 3} =1}, but \m {\i \SB {5, 5; 4, 6} =0}.

Formally,
\Disp {
\NC \NC \i \SB {
x_{1,1}, x_{1,2}, \ldots, x_{1,M\SB{1}} \mid
\ldots \mid
x_{N,1}, x_{N,2}, \ldots, x_{N,M\SB{N}}
} \NR
\NC =\NC \startcases
\NC 1, \MC x_{1,1} =\ldots =x_{1,M\SB{1}}
   \neq \dots
   \neq x_{N,1} =\ldots =x_{N,M\SB{N}} \NR
\NC 0, \NC \Q \Rm {otherwise} \NR
\stopcases \NR[+]
}

If we introduce
\Disp{
\NC \NC I_2 \SB {x_1, y_1, x_2, y_2} \NR
\NC =\NC \i \SB {x_1, x_2} \D \i \SB {y_1, y_2} \NR[+]
\NC \NC I_4 \SB {x_1, y_1, x_2, y_2, x_3, y_3, x_4, y_4} \NR
\NC =\NC \i \SB {x_1, x_2, x_3, x_4} \D \i \SB {y_1, y_2, y_3, y_4} \NR[+]
\NC \NC I_{2,2} \SB {x_1, y_1, x_2, y_2, x_3, y_3, x_4, y_4} \NR
\NC =\NC \RB {\i \SB {x_1, x_2 \mid x_3, x_4} +\i \SB {x_1, x_3 \mid x_2, x_4}} \i \SB {y_1, y_2, y_3, y_4} \NR
\NC \NC \FourQ + \i \SB {x_1, x_2, x_3, x_4} \RB {\i \SB {y_1, y_2 \mid y_3, y_4} +\i \SB {y_1, y_3 \mid y_2, y_4}} \NR
\NC \NC \FourQ + \i \SB {x_1, x_2 \mid x_3, x_4} \i \SB {y_1, y_2 \mid y_3, y_4} \NR
\NC \NC \FourQ + \i \SB {x_1, x_4 \mid x_2, x_3} \i \SB {y_1, y_4 \mid y_2, y_3} \NR[+]
}

Clearly, by construction,
\Disp {
\NC \NC \MB{E} \SB {\M{F}_B ^\ast \DB {n_R, n_Y}  \M{F}_B \DB {n_R', n_Y'}} \NR
\NC = \NC \MB{E} \SB {\M{W}_B ^\Tr \DB {n_R, n_Y}  \M{W}_B ^\Adj \DB {n_R', n_Y'}} \NR
\NC = \NC I_2 \SB {n_R, n_Y, n_R', n_Y'} \D M_{B,2}, \NR[+]
%
\NC \NC \MB{E} \SB {\M{F}_R ^\ast \DB {n_H, n_R}  \M{F}_R \DB {n_H', n_R'}} \NR
\NC = \NC \MB{E} \SB {\M{W}_R ^\Tr \DB {n_H, n_R}  \M{W}_R ^\Adj \DB {n_H', n_R'}} \NR
\NC = \NC I_2 \SB {n_H, n_R, n_H', n_R'} \D M_{R,2}, \NR[+]
%
\NC \NC \MB{E} \SB {\M{F}_B ^\ast \DB {n_R, n_Y}  \M{F}_B ^\ast \DB {n_R', n_Y'}  \M{F}_B \DB {n_R'', n_Y''} \M{F}_B \DB {n_R''', n_Y'''}} \NR
\NC = \NC \MB{E} \SB {\M{W}_B ^\Tr \DB {n_R, n_Y}  \M{W}_B ^\Tr \DB {n_R', n_Y'}  \M{W}_B ^\Adj \DB {n_R'', n_Y''} \M{W}_B ^\Adj \DB {n_R''', n_Y'''}}, \NR
\NC = \NC I_4 \SB {n_R, n_Y, n_R', n_Y', n_R'', n_Y'', n_R''', n_Y'''} \D M_{B,4} \NR
\NC \NC \FourQ
I_{2,2} \SB {n_R, n_Y, n_R', n_Y', n_R'', n_Y'', n_R''', n_Y'''} \D M_{B,2}^2 \NR[+]
%
\NC \NC \MB{E} \SB {\M{F}_R ^\ast \DB {n_H, n_R}  \M{F}_R ^\ast \DB {n_H', n_R'}  \M{F}_R \DB {n_H'', n_R''} \M{F}_R \DB {n_H''', n_R'''}} \NR
\NC = \NC \MB{E} \SB {\M{W}_R ^\Tr \DB {n_H, n_R}  \M{W}_R ^\Tr \DB {n_H', n_R'}  \M{W}_R ^\Adj \DB {n_H'', n_R''} \M{W}_R ^\Adj \DB {n_H''', n_R'''}}, \NR
\NC = \NC I_4 \SB {n_H, n_R, n_H', n_R', n_H'', n_R'', n_H''', n_R'''} \D M_{R,4} \NR
\NC \NC \FourQ
I_{2,2} \SB {n_H, n_R, n_H', n_R', n_H'', n_R'', n_H''', n_R'''} \D M_{R,2}^2 \NR[+]
%
\NC n_Y, n_Y'
= \NC 0, 1, 2, \ldots, N_Y -1 \NR
\NC n_R, n_R'
= \NC 0, 1, 2, \ldots, N_R -1 \NR
\NC n_H, n_H'
= \NC 0, 1, 2, \ldots, N_H -1 \NR
}
It will be similarly understood below that \m {n_Y, n_R, n_H} (and primed variables) run through all possible values.

Now, denote for short
\Disp {
\NC \M{F} :=\NC \M {F}_R \M {F}_B \NR[+]
\NC \M{W} :=\NC \M {W}_B \M {W}_R \NR[+]
}
Then, according to definition
\Disp {
\NC \NC \M{F} ^\ast \DB {n_H, n_Y} 
\M{F} \DB {n_H', n_Y'} \NR
\NC =\NC \sum_{n_R, n_R' =0}^{N_R-1}
\M{F}_R ^\ast \DB {n_H, n_R} 
\M{F}_B ^\ast \DB {n_R, n_Y} 
\M{F}_R \DB {n_H', n_R'}
\M{F}_B \DB {n_R', n_Y'} \NR[+]
}
By above,
\Disp {
\NC \NC \MB{E} \SB {\M{F} ^\ast \DB {n_H, n_Y}  \M{F} \DB {n_H', n_Y'}} \NR
\NC = \NC \MB{E} \SB {\M{W}^\Tr \DB {n_Y, n_H}  \M{W}^\Adj \DB {n_Y', n_H'}} \NR
\NC = \NC I_2 \SB {n_H, n_Y, n_H', n_Y'} \D N_R M_{B,2} M_{R,2} \NR[+]
}

By the same token,
\Disp {
\NC \NC \M{F} ^\ast \DB {n_H, n_Y} 
\M{F} ^\ast \DB {n_H', n_Y'} 
\M{F} \DB {n_H'', n_Y''}
\M{F} \DB {n_H''', n_Y'''} \NR
\NC =\NC \sum_{n_R, n_R', n_R'', n_R''' =0}^{N_R-1}
\M{F}_R ^\ast \DB {n_H, n_R} 
\M{F}_B ^\ast \DB {n_R, n_Y} 
\M{F}_R ^\ast \DB {n_H', n_R'} 
\M{F}_B ^\ast \DB {n_R', n_Y'}  \NR
\NC \NC \SixQ \M{F}_R \DB {n_H'', n_R''}
\M{F}_B \DB {n_R'', n_Y''}
\M{F}_R \DB {n_H''', n_R'''}
\M{F}_B \DB {n_R''', n_Y'''} \NR[+]
}
And it can be seen that
\Disp {
\NC \NC \MB{E} \SB {
   \M{F} ^\ast \DB {n_H, n_Y} \M{F} ^\ast \DB {n_H', n_Y'} 
   \M{F} \DB {n_H'', n_Y''} \M{F} \DB {n_H''', n_Y'''}
} \NR
\NC =\NC \MB{E} \SB {
   \M{W} ^\Tr \DB {n_H, n_Y} \M{W} ^\Tr \DB {n_H', n_Y'} 
   \M{W} ^\Adj \DB {n_H'', n_Y''} \M{W} ^\Adj \DB {n_H''', n_Y'''}
} \NR
\NC = \NC I_4 \SB {n_H, n_Y, n_H', n_Y', n_H'', n_Y'', n_H''', n_Y'''}
\D \RB {N_R M_{R,4} M_{B,4} +N_R \RB {N_R-1} M_{R,2}^2 M_{B,2}^2} \NR
\NC \NC \FourQ
I_{2,2} \SB {n_H, n_Y, n_H', n_Y', n_H'', n_Y'', n_H''', n_Y'''}
\D \F{1}{2} N_R \RB {N_R+1} M_{R,2}^2 M_{B,2}^2 \NR[+]
}

Furthermore, by the nature of Kronecker product, and the symmetry between \m {\M{F} ^\Tr} and \m {\M{W}}, we realize that the expectation of such product of entries of \m {\M{Q}} is just square of the product in such manner of entries of \m {\M{F} ^\Tr} (or \m {\M{W}}).
\Result
{Lemma}
{
\Disp {
\NC \NC \MB{E} \SB {\M{Q} ^\ast \DB {n_y, n_h}  \M{Q} \DB {n_y', n_h'}} \NR
\NC = \NC I_2 \SB {n_y, n_h, n_y', n_h'} \D N_R^2 M_{B,2}^2 M_{R,2}^2 \NR[+]
}
}
And
\Result
{Lemma}
{
\Disp {
\NC \NC \MB{E} \SB {
   \M{Q} ^\ast \DB {n_y, n_h} \M{Q} ^\ast \DB {n_y', n_h'} 
   \M{Q} \DB {n_y'', n_h''} \M{Q} \DB {n_y''', n_h'''}
} \NR
\NC = \NC I_4 \SB {n_y, n_h, n_y', n_h', n_y'', n_h'', n_y''', n_h'''} \D N_R^2 M_{R,4}^2 M_{B,4}^2 \NR
\NC \NC \FourQ
I_{2,2} \SB {n_y, n_h, n_y', n_h', n_y'', n_h'', n_y''', n_h'''} \D N_R^4 M_{R,2}^4 M_{B,2}^4 \NR[+]
}
}

Now, fix any test vector \m {\V{u} \in \MB{V}_\MB{C} \SB{N_Y}}, we shall study the probability concentration of
\Disp {
\NC \rho \SB {\o_t, \o_r}
:=\NC \VNm {\M{Q} \SB {\o_t, \o_r} \V{u}}_2^2 \NR[+]
}
by exploiting Chernoff Inequality.

In this section, since the quantity \m {\Nm {\rho -\MB{E} \SB {\rho}}} is linear in \m {\VNm {\V{u}} _2}, by a scaling argument we may set, without loss of generality,
\Disp {
\NC \VNm {\V{u}} _2
=\NC 1 \NR[+]
}

\stopsubsection

\startsubsection [title={The First Moment}]

The Chernoff Inequality is most effective if all higher moments are known.
To begin with, study \m {\MB{E} \SB {\rho \SB {\o_t, \o_r}}}.
The calculation can be simplified if we expand the square of norm and pass through the expectation sign.
\Disp{
\NC \NC \MB{E} \SB {\rho \SB {\o_t, \o_r}} \NR
\NC = \NC \MB{E} \SB {
  \sum_{n_y =0}^{N_y-1}
  \Nm {\sum_{n_h =0}^{N_h-1}
  \M{Q} ^\ast \DB {n_y, n_h}  \V{u} \DB{n_h}} ^2
} \NR
\NC = \NC
\MB{E} \SB {
  \sum_{n_y =0}^{N_y-1}
  \sum_{n_h, n_h' =0}^{N_h-1}
  \M{Q} ^\ast \DB {n_y, n_h} \V{u} \DB{n_h}
  \M{Q} \DB {n_y, n_h'}  \V{u} ^\ast \DB{n_h'}
} \NR
\NC = \NC
\sum_{n_y =0}^{N_y-1}
\sum_{n_h, n_h' =0}^{N_h-1}
\MB{E} \SB {\M{Q} ^\ast \DB {n_y, n_h} \M{Q} \DB {n_y, n_h'} }
\V{u} \DB{n_h} \V{u} ^\ast \DB{n_h'} \NR
\NC = \NC N_R^2 M_{B,2}^2 M_{R,2}^2 N_y
\sum_{n_h, n_h' =0}^{N_h-1}
\i \SB {n_h, n_h'} \D \V{u} \DB{n_h} \V{u} ^\ast \DB{n_h'} \NR
\NC = \NC N_R^2 M_{B,2}^2 M_{R,2}^2 N_y \D \VNm {\V{u}} _2 ^2 \NR[+]
}
Thus, we may set
\Disp{
\NC \l_B
=\NC \F {1} {\R {N_Y}} \NR[+]
\NC \l_R
=\NC \F {1} {\R {N_R}} \NR[+]
}
To make
\Disp{
\NC \MB{E} \SB {\rho}
=\NC \VNm {\V{u}} _2 ^2
}

\stopsubsection

\startsubsection [title={The Second Moment}]

Similarly, we have to find, in advance, the second moment.
Pay attention to the order of indices in the product of entries of \m {\V{u}} and that of \m {\M{Q}}.
\Disp{
\NC \NC \MB{E} \SB {\rho \SB {\o_t, \o_r} ^2} \NR
%
\NC =\NC \MB{E} \SB {
  \RB {
    \sum_{n_R =0}^{N_R-1}
    \Nm {\sum_{n_Y =0}^{N_Y-1} \M{Q} ^\ast \DB {n_R, n_Y}  \V{u} \DB{n_Y}} ^2
  } ^2
} \NR
%
\NC = \NC
\MB{E} \Bigg[
  \sum_{n_y, n_y =0}^{N_y-1}
  \sum_{n_h, n_h', n_h'', n_h''' =0}^{N_h-1}
  \M{Q} ^\ast \DB {n_y, n_h} \M{Q} \DB {n_y, n_h'}
  \V{u} \DB{n_h} \V{u} ^\ast \DB{n_h'} \NR
\NC \NC \SixQ
  \M{Q} ^\ast \DB {n_y', n_h''} \M{Q} \DB {n_y', n_h'''}
  \V{u} \DB{n_h''} \V{u} ^\ast \DB{n_h'''}
\Bigg] \NR
%
\NC = \NC
  \sum_{n_y, n_y' =0}^{N_y-1}
  \sum_{n_h, n_h', n_h'', n_h''' =0}^{N_h-1}
  \MB{E} \SB{
    \M{Q} ^\ast \DB {n_y, n_h} \M{Q} ^\ast \DB {n_y', n_h''}
    \M{Q} \DB {n_y, n_h'} \M{Q} \DB {n_y', n_h'''}
  } \NR
\NC \NC \SixQ
\D \V{u} ^\ast \DB{n_h'} \V{u} ^\ast \DB{n_h'''}
\V{u} \DB{n_h} \V{u} \DB{n_h''} \NR
%
\NC = \NC
\sum_{n_y, n_y' =0}^{N_y-1}
\sum_{n_h, n_h', n_h'', n_h''' =0}^{N_h-1}
  \Big(
    I_4 \SB {n_y, n_h, n_y', n_h'', n_y, n_h', n_y', n_h'''} \D N_R^2 M_{R,4}^2 M_{B,4}^2 \NR
\NC \NC \SixQ
    +I_{2,2} \SB {n_y, n_h, n_y', n_h'', n_y, n_h', n_y', n_h'''} \D N_R^4 M_{R,2}^4 M_{B,2}^4
  \Big) \NR
\NC \NC \SixQ
  \D \V{u} ^\ast \DB{n_h'} \V{u} ^\ast \DB{n_h'''}
  \V{u} \DB{n_h} \V{u} \DB{n_h''} \NR
%
\NC = \NC N_R^2 M_{R,4}^2 M_{B,4}^2 \NR
\NC \NC \SixQ \D \sum_{n_y, n_y' =0}^{N_y-1}
\sum_{n_h, n_h', n_h'', n_h''' =0}^{N_h-1}
   \i \SB {n_y, n_y'} \i \SB {n_h, n_h'', n_h', n_h'''} \NR
  \NC \NC \SixQ \D \V{u} ^\ast \DB{n_h'} \V{u} ^\ast \DB{n_h'''}
  \V{u} \DB{n_h} \V{u} \DB{n_h''} \NR
\NC \NC \FourQ + N_R^4 M_{R,2}^4 M_{B,2}^4
\sum_{n_y, n_y' =0}^{N_y-1}
\sum_{n_h, n_h', n_h'', n_h''' =0}^{N_h-1} \NR
\NC \NC \SixQ \Bigg(
  \i \SB {n_y, n_y'} \i \SB {n_h, n_h'' \mid n_h', n_h'''}
  +\i \SB {n_y, n_y'} \i \SB {n_h, n_h' \mid n_h'', n_h'''} \NR
  \NC \NC \SixQ +\i \SB {n_y \mid n_y'} \i \SB {n_h, n_h', n_h'', n_h'''}
  +\i \SB {n_y \mid n_y'} \i \SB {n_h, n_h' \mid n_h'', n_h'''}
\Bigg) \NR
  \NC \NC \SixQ \D \V{u} ^\ast \DB{n_h'} \V{u} ^\ast \DB{n_h'''}
  \V{u} \DB{n_h} \V{u} \DB{n_h''} \NR
%
\NC = \NC N_R^2 M_{R,4}^2 M_{B,4}^2
\D N_y \sum_{n_h =0}^{N_h-1} \Nm {\V{u} \DB{n_h}} ^4 \NR
\NC \NC \FourQ
+ N_R^4 M_{R,2}^4 M_{B,2}^4 \NR
\NC \NC \SixQ \D \Bigg(
  N_y \sum_{n_h, n_h' =0}^{N_h-1} \i \SB {n_h \mid n_h'} \V{u} ^\ast \DB{n_h} ^2 \V{u} \DB{n_h'} ^2 \NR
  \NC \NC \SixQ +N_y \sum_{n_h, n_h' =0}^{N_h-1} \i \SB {n_h \mid n_h'} \Nm {\V{u} \DB{n_h}} ^2 \Nm {\V{u} \DB{n_h'}} ^2 \NR
  \NC \NC \SixQ +\RB {N_y^2 -N_y} \sum_{n_h =0}^{N_h-1} \Nm {\V{u} \DB{n_h}} ^4 \NR
  \NC \NC \SixQ +\RB {N_y^2 -N_y} \sum_{n_h, n_h' =0}^{N_h-1} \i \SB {n_h \mid n_h'} \Nm {\V{u} \DB{n_h}} ^2 \Nm {\V{u} \DB{n_h'}} ^2
\Bigg) \NR[+]
}

The third term is \m {\VNm {\V{u}} _4 ^4}.
The second and fourth terms are exactly \m {\VNm {\V{u}} _2 ^4 -\VNm {\V{u}} _4 ^4}.
The first term in the big parentheses can also be bounded by triangle inequality to be \m {\VNm {\V{u}} _2 ^4 -\VNm {\V{u}} _4 ^4}.
\Disp {
\NC \NC \sum_{n_h, n_h' =0}^{N_h-1}
\i \SB {n_h \mid n_h'} \V{u} ^\ast \DB{n_h} ^2 \V{u} \DB{n_h'} ^2 \NR
\NC \leq \NC \Nm {\sum_{n_h =0}^{N_h-1} \V{u} ^\ast \DB{n_h} ^2} ^2
-\sum_{n_h =0}^{N_h-1} \V{u} ^\ast \DB{n_h} ^4 \NR
\NC \leq \NC \VNm {\V{u}} _2 ^4 -\VNm {\V{u}} _4 ^4 \NR[+]
}

Thus,
\Disp {
\NC \NC \MB{E} \SB {\rho ^2} \NR
\NC = \NC N_R^2 M_{R,4}^2 M_{B,4}^2
\D N_y \VNm {\V{u}} _4 ^4 \NR
\NC \NC \FourQ + \RB {N_R^2 M_{R,2}^2 M_{B,2}^2} ^2 \NR
\NC \NC \SixQ \Bigg(
  N_y \RB {\VNm {\V{u}} _2 ^4 -\VNm {\V{u}} _4 ^4}
  +N_y \RB {\VNm {\V{u}} _2 ^4 -\VNm {\V{u}} _4 ^4} \NR
  \NC \NC \SixQ +\RB {N_y^2 -N_y} \VNm {\V{u}} _4 ^4
  +\RB {N_y^2 -N_y} \RB {\VNm {\V{u}} _2 ^4 -\VNm {\V{u}} _4 ^4}
\Bigg) \NR
%
\NC =\NC \RB {1 +N_Y^{-2}} \VNm {\V{u}} _2 ^4
+\RB {-2 N_Y^2 -2 \RB {1 -2N_R^{-2}} N_Y^{-2}} \VNm {\V{u}} _4 ^4 \NR
\NC \leq \NC \RB {1 +N_Y^{-2}} \VNm {\V{u}} _2 ^4 \NR[+]
}
Noting that \m {N_H \gg N_R \gg N_Y \gg 1}, and the fact \m {\VNm {\V{u}} _4 \geq N_H^{-1/2} \VNm {\V{u}} _2}, we see that it is not a big deal to drop the \m {\VNm {\V{u}} _4} term .

\stopsubsection

\startsubsection [title={The Higher Moments}]

In bounding the higher moments of \m {\rho}, let us be more generous.
Fix \m {k =3, 4, 5, \dots}.
If we introduce the change of variables
\Disp {
\NC l_Y :=\NC \Fl {n_Y /N_Y} \NR
\NC m_Y :=\NC n_Y \; \Rm{Mod}\; N_Y \NR
\NC l_H :=\NC \Fl {n_H /N_H} \NR
\NC l_H' :=\NC \Fl {n_H' /N_H} \NR
\NC m_H :=\NC n_H \; \Rm{Mod}\; N_H \NR
\NC m_H' :=\NC n_H' \; \Rm{Mod}\; N_H \NR
}
By rearranging,
\Disp{
  \NC \NC \MB{E} \SB {\rho \SB {\o_t, \o_r} ^k} \NR
%
  \NC =\NC \MB{E} \bigg[
    \bigg|
      \sum_{l_Y, m_Y =0}^{N_Y-1}
        \sum_{n_R, n_R', n_R'', n_R''' =0}^{N_R-1}
          \sum_{l_H, l_H', m_H, m_H' =0}^{N_H-1} \NR
            \NC \NC \FourQ
            \M{F}_B ^{\Adj} \DB {l_Y, n_R}
            \M{F}_R ^{\Adj} \DB {n_R, l_H}
            \M{W}_B ^{\ast} \DB {m_Y, n_R'}
            \M{W}_R ^{\ast} \DB {n_R', m_H} \NR
            \NC \NC \FourQ
            \M{F}_B ^{\Tr} \DB {l_Y, n_R''}
            \M{F}_R ^{\Tr} \DB {n_R'', l_H'}
            \M{W}_B \DB {m_Y, n_R'''}
            \M{W}_R \DB {n_R''', m_H'} \NR
            \NC \NC \FourQ
            \V{u} \DB{l_H N_H +m_H}
            \V{u} ^\ast \DB{l_H' N_H +m_H'}
    \bigg| ^k
  \bigg] \NR
%
  \NC =\NC \MB{E} \Bigg[
    \Bigg|
      \sum_{l_H, l_H', m_H, m_H' =0}^{N_H-1}
        \V{u} \DB{l_H N_H +m_H}
        \V{u} ^\ast \DB{l_H' N_H +m_H'} \NR
        \NC \NC \FourQ \D \sum_{l_Y =0}^{N_Y-1}
          \RB{
            \sum_{n_R =0}^{N_R-1}
              \M{F}_B ^{\Adj} \DB {l_Y, n_R}
              \M{F}_R ^{\Adj} \DB {n_R, l_H}
          }
          \RB{
            \sum_{n_R =0}^{N_R-1}
              \M{F}_B ^{\Tr} \DB {l_Y, n_R}
              \M{F}_R ^{\Tr} \DB {n_R, l_H'}
          }
        \NR
      \NC \NC \FourQ \D \sum_{m_Y =0}^{N_Y-1}
        \RB{
          \sum_{n_R =0}^{N_R-1}
            \M{W}_B ^{\ast} \DB {m_Y, n_R}
            \M{W}_R ^{\ast} \DB {n_R, m_H}
        }
        \RB{
          \sum_{n_R =0}^{N_R-1}
            \M{W}_B \DB {m_Y, n_R}
            \M{W}_R \DB {n_R, m_H'}
        }
    \Bigg| ^k
  \Bigg] \NR
}
%
We see that, by absorbing the phase into entry of \m {\M{F}_R} (resp.\ \m {\M{W}_R}), without loss of generality we may assume each entry of \m {\M{F}_B} (resp.\ \m {\M{W}_B}) to be positive, and also assume each copmonent of \m {\V {u}} to be positive.

By Jensen's Inequality;
by probability independence;
by Power Mean Inequality to replace \m {l_H} and \m {l_H'} by the same number (resp.\ \m {m_H} and \m {m_H'});
by arranging;
by probability independence,
\Disp{
  \NC \leq \NC \Nm{
    \sum_{l_H, l_H', m_H, m_H' =0}^{N_H-1}
      \V{u} \DB{l_H N_H +m_H}
      \V{u} ^\ast \DB{l_H' N_H +m_H'}
  } ^{k-1} \NR
  \NC \NC \Q \Q \D \sum_{l_H, l_H', m_H, m_H' =0}^{N_H-1}
    \V{u} \DB{l_H N_H +m_H}
    \V{u} ^\ast \DB{l_H' N_H +m_H'} \NR
  \NC \NC \FourQ \D \MB{E} \Bigg[
    \Bigg|
      \sum_{l_Y =0}^{N_Y-1}
      \RB{
        \sum_{n_R =0}^{N_R-1}
          \M{F}_B ^{\Adj} \DB {l_Y, n_R}
          \M{F}_R ^{\Adj} \DB {n_R, l_H}
      }
      \RB{
        \sum_{n_R =0}^{N_R-1}
          \M{F}_B ^{\Tr} \DB {l_Y, n_R}
          \M{F}_R ^{\Tr} \DB {n_R, l_H'}
      }
      \Bigg| ^k \NR
    \NC \NC \FourQ \Bigg|
      \sum_{m_Y =0}^{N_Y-1}
      \RB{
        \sum_{n_R =0}^{N_R-1}
          \M{W}_B ^{\ast} \DB {m_Y, n_R}
          \M{W}_R ^{\ast} \DB {n_R, m_H}
      }
      \RB{
        \sum_{n_R =0}^{N_R-1}
          \M{W}_B \DB {m_Y, n_R}
          \M{W}_R \DB {n_R, m_H'}
      }
    \Bigg| ^k
  \Bigg] \NR
%
  \NC =\NC \Nm{
    \sum_{l_H, l_H', m_H, m_H' =0}^{N_H-1}
      \V{u} \DB{l_H N_H +m_H}
      \V{u} ^\ast \DB{l_H' N_H +m_H'}
  } ^{k-1} \NR
  \NC \NC \Q \Q \D \sum_{l_H, l_H', m_H, m_H' =0}^{N_H-1}
    \V{u} \DB{l_H N_H +m_H}
    \V{u} ^\ast \DB{l_H' N_H +m_H'} \NR
  \NC \NC \FourQ \D \MB{E} \Bigg[
    \Bigg|
      \sum_{l_Y =0}^{N_Y-1}
      \RB{
        \sum_{n_R =0}^{N_R-1}
          \M{F}_B ^{\Adj} \DB {l_Y, n_R}
          \M{F}_R ^{\Adj} \DB {n_R, l_H}
      }
      \RB{
        \sum_{n_R =0}^{N_R-1}
          \M{F}_B ^{\Tr} \DB {l_Y, n_R}
          \M{F}_R ^{\Tr} \DB {n_R, l_H'}
      }
    \Bigg| ^k
  \Bigg] \NR
  \NC \NC \FourQ \D \MB{E} \Bigg[
    \Bigg|
      \sum_{m_Y =0}^{N_Y-1}
      \RB{
        \sum_{n_R =0}^{N_R-1}
          \M{W}_B ^{\ast} \DB {m_Y, n_R}
          \M{W}_R ^{\ast} \DB {n_R, m_H}
      }
      \RB{
        \sum_{n_R =0}^{N_R-1}
          \M{W}_B \DB {m_Y, n_R}
          \M{W}_R \DB {n_R, m_H'}
      }
    \Bigg| ^k
  \Bigg] \NR
%
  \NC \leq \NC \Nm{
    \sum_{l_H, l_H', m_H, m_H' =0}^{N_H-1}
      \V{u} \DB{l_H N_H +m_H}
      \V{u} ^\ast \DB{l_H' N_H +m_H'}
  } ^{k-1} \NR
  \NC \NC \Q \Q \D \sum_{l_H, l_H', m_H, m_H' =0}^{N_H-1}
    \V{u} \DB{l_H N_H +m_H}
    \V{u} ^\ast \DB{l_H' N_H +m_H'} \NR
  \NC \NC \FourQ \D \MB{E} \Bigg[
    \Bigg|
      \sum_{l_Y =0}^{N_Y-1}
      \RB{
        \sum_{n_R =0}^{N_R-1}
          \M{F}_B ^{\Adj} \DB {l_Y, n_R}
          \M{F}_R ^{\Adj} \DB {n_R, 0}
      }
      \RB{
        \sum_{n_R =0}^{N_R-1}
          \M{F}_B ^{\Tr} \DB {l_Y, n_R}
          \M{F}_R ^{\Tr} \DB {n_R, 0}
      }
    \Bigg| ^k
  \Bigg] \NR
  \NC \NC \FourQ \D \MB{E} \Bigg[
    \Bigg|
      \sum_{m_Y =0}^{N_Y-1}
      \RB{
        \sum_{n_R =0}^{N_R-1}
          \M{W}_B ^{\ast} \DB {m_Y, n_R}
          \M{W}_R ^{\ast} \DB {n_R, 0}
      }
      \RB{
        \sum_{n_R =0}^{N_R-1}
          \M{W}_B \DB {m_Y, n_R}
          \M{W}_R \DB {n_R, 0}
      }
    \Bigg| ^k
  \Bigg] \NR
%
  \NC =\NC \Nm{
    \sum_{l_H, l_H', m_H, m_H' =0}^{N_H-1}
      \V{u} \DB{l_H N_H +m_H}
      \V{u} ^\ast \DB{l_H' N_H +m_H'}
  } ^{k} \NR
  \NC \NC \FourQ \D \MB{E} \Bigg[
    \Bigg|
      \sum_{l_Y =0}^{N_Y-1}
      \bigg|
        \sum_{n_R =0}^{N_R-1}
          \M{F}_B ^{\Tr} \DB {l_Y, n_R}
          \M{F}_R ^{\Tr} \DB {n_R, 0}
      \bigg| ^2
    \Bigg| ^k
  \Bigg] \NR
  \NC \NC \FourQ \D \MB{E} \Bigg[
    \Bigg|
      \sum_{m_Y =0}^{N_Y-1}
      \bigg|
        \sum_{n_R =0}^{N_R-1}
          \M{W}_B \DB {m_Y, n_R}
          \M{W}_R \DB {n_R, 0}
      \bigg| ^2
    \Bigg| ^k
  \Bigg] \NR
%
  \NC =\NC \Nm{
    \sum_{l_H, l_H', m_H, m_H' =0}^{N_H-1}
      \V{u} \DB{l_H N_H +m_H}
      \V{u} ^\ast \DB{l_H' N_H +m_H'}
  } ^{k} \NR
  \NC \NC \SixQ \D \MB{E} \Bigg[
    \RB{
      \sum_{l_Y =0}^{N_Y-1}
      \bigg|
        \sum_{n_R =0}^{N_R-1}
          \M{F}_B ^{\Tr} \DB {l_Y, n_R}
          \M{F}_R ^{\Tr} \DB {n_R, 0}
      \bigg| ^2
    } ^k
  \Bigg]^2 \NR
}
By Cauchy Inequality and by assumption that \m {\V {u}} has unity norm;
by Jensen Inequality;
by i.i.d.\ of each quantity that contains \m {n_Y};
by spliting real and imaginary parts;
by Jensen Inequality (or Power Mean Inequality);
by arranging;
\Disp{
  \NC \leq \NC \VNm {\V {u}} _2 ^{2k}
  \D \MB{E} \Bigg[
    \RB{
      \sum_{l_Y =0}^{N_Y-1}
      \bigg|
        \sum_{n_R =0}^{N_R-1}
          \M{F}_B \DB {l_Y, n_R}
          \M{F}_R \DB {n_R, 0}
      \bigg| ^2
    } ^k
  \Bigg]^2 \NR
%
  \NC \leq \NC 1 \D N_Y ^{2k}
  \MB{E} \Bigg[
    N_Y ^{-1}
    \sum_{l_Y =0}^{N_Y-1}
    \bigg|
      \sum_{n_R =0}^{N_R-1}
        \M{F}_B \DB {0, n_R}
        \M{F}_R \DB {n_R, 0}
    \bigg| ^{2k}
  \Bigg]^2 \NR
%
  \NC =\NC N_Y ^{2k}
  \MB{E} \Bigg[
    \bigg|
      \sum_{n_R =0}^{N_R-1}
        \M{F}_B \DB {0, n_R}
        \M{F}_R \DB {n_R, 0}
    \bigg| ^{2k}
  \Bigg]^2 \NR
%
  \NC =\NC N_Y ^{2k}
  \MB{E} \bigg[
    \bigg|
      \RB{
        \sum_{n_R =0}^{N_R-1}
          \M{F}_B ^{\Tr} \DB {0, n_R}
          \MF{Re} \M{F}_R ^{\Tr} \DB {n_R, 0}
      }^2 \NR
      \NC \NC \SixQ +\RB{
        \sum_{n_R =0}^{N_R-1}
          \M{F}_B ^{\Tr} \DB {0, n_R}
          \MF{Im} \M{F}_R ^{\Tr} \DB {n_R, 0}
      }^2
    \bigg| ^{k}
  \bigg]^2 \NR
%
  \NC \leq \NC 2^{2k} N_Y ^{2k}
  \MB{E} \bigg[
    \bigg|
      \sum_{n_R =0}^{N_R-1}
        \M{F}_B ^{\Tr} \DB {0, n_R}
        \MF{Re} \M{F}_R ^{\Tr} \DB {n_R, 0}
    \bigg| ^{2k}
  \bigg]^2 \NR
%
  \NC =\NC \MB{E} \bigg[
    \bigg|
      \R{N_R} ^{-1/2} \sum_{n_R =0}^{N_R-1}
        \R{2N_Y} \M{F}_B ^{\Tr} \DB {0, n_R}
        \R{N_R} \MF{Re} \M{F}_R ^{\Tr} \DB {n_R, 0}
    \bigg| ^{2k}
  \bigg]^2 \NR
}

Suppose (now use \m {n} in place of \m {n_R} for short)
\Disp{
  \NC \a_{n}
  :=\NC \Nm {
    \R{2N_Y} \M{F}_B ^{\Tr} \DB {0, n}
    \R{N_R} \MF{Re} \M{F}_R ^{\Tr} \DB {n, 0}
  } \NR[+]
  \NC \b_{n}
  :=\NC \Ss{sgn} \SB{
    \R{N_R} \MF{Re} \M{F}_R ^{\Tr} \DB {n, 0}
  } \NR[+]
}
Then \m {\b_{n}} takes value \m {\pm 1}, and is Bernoulli with probability \m {1/2} of each.

To find the moment of \m {\a_{n}}, the \m {\M{F}_B} factor follows Rayleigh distribution, and its moments can be looked up.
And as for the \m {\M{F}_R} factor, it is just
\Disp{
\NC \NC \F {1} {2\pi} \int _{x=0} ^{2\pi}
\RB {\cos ^k x} \Ss{d} x \NR
\NC =\NC \F {\RB {q-1} !!} {q !!} \D
\startcases
\MC 2/\pi, \MC \m{q =0,2,4,\dots} \NR
\MC 1, \MC \m{q =1,3,5,\dots} \NR
\stopcases \NR
}
So,
\Disp{
\NC \MB{E} \SB {\Nm {\a_{n}} ^k}
=\NC \G \SB {\F{k}{2} +1}
\D \F{1} {\R{\pi}} \F{\G \SB {\F{k}{2} +\F{1}{2}}} {\G \SB {\F{k}{2} +1}} \NR
\NC =\NC \G \SB {\F{k}{2} +\F{1}{2}} \NR[+]
}
Introduce \m {\g_{n}}, i.i.d.\ Standard Normal r.vs.
Now, for \m {k =1,2,3,\dots}, we have the neat relation
\Disp{
\NC \MB{E} \SB {\Nm {\a_{n}} ^k}
\leq \NC \MB{E} \SB {\Nm {\g_{n}} ^k}
}

\Result
{Lemma}
{
Let \m {\a_{n}' \geq 0} and \m {\g_{n}' \geq 0} be i.i.d.\ random variables,
and \m {\b_{n}} be Bernoulli r.vs.\ that take value \m {\pm 1} with probability \m {1/2} of each.
Any of them are independent.
If, for \m {k =1,2,3,\dots},
\Disp{
\NC \MB{E} \SB {\Nm {\a_{n}'} ^k}
\leq \NC \MB{E} \SB {\Nm {\g_{n}'} ^k} \NR[+]
}
Then, for \m {k =1,2,3,\dots},
\Disp{
\NC \MB{E} \SB {\Nm {\sum_{n=0}^{N-1} \a_{n}' \b_{n}} ^k}
\leq \NC \MB{E} \SB {\Nm {\sum_{n=0}^{N-1} \g_{n}' \b_{n}} ^k} \NR[+]
}
}

The proof of even \m {k} is straightforward, by expanding and comparing term-by-term.
The proof of odd \m {k} also follows by observation \m {\RB {x} ^k =\RB {\RB {x} ^{2k}} ^{1/2}} and Generalized Binomial Theorem for power \m {1/2}.


Then, since the linear combination of Normal r.vs.\ is also a Normal r.v.,
\Disp{
  \NC \NC \MB{E} \SB {\rho ^k} \NR
  \NC \leq \NC \MB{E} \bigg[
    \bigg|
      \R{N_R} ^{-1/2} \sum_{n =0}^{N_R-1}
      \g_{n}
    \bigg| ^{2k}
  \bigg]^2 \NR
  \NC \leq \NC \G \SB {k +1}^2 \NR[+]
}

\stopsubsection

\startsubsection [title={Using Chernoff Inequality}]

Imitating the Chernoff Inequality argument in classical probability, we have, for any \m {s_+ >0} so that, supposedly, both sides \m {< \infty},
\Disp {
\NC p
= \NC \MB{P} \SB {\rho \geq \RB {1 +\e} \MB{E} \SB {\rho}} \NR
\NC =\NC \MB{P} \SB {\rho \geq 1 +\e} \NR
\NC =\NC \MB{P} \SB {\R {\rho} \geq \R {1 +\e}} \NR
\NC =\NC \MB{P} \SB {\Ss {e} ^{s_+ \R {\rho}} \geq \Ss {e} ^{s_+ \R {1 +\e}}} \NR
\NC \leq \NC \Ss {e} ^{-s_+ \R {1 +\e}} \MB{E} \SB {\Ss {e} ^{s_+ \R {\rho}}} \NR[+]
}
Assuming the convergence of power series of \m {\Ss {e} ^{s_+ \R {\rho}}} for a moment, which will be justified below,
\Disp {
\NC \NC \MB{E} \SB {\Ss {e} ^{s_+ \R {\rho}}} \NR
%
\NC \leq \NC 1
+\MB{E} \SB {s_+ \R {\rho}}
+\MB{E} \SB {\F{s_+ ^2} {2} \rho}
+\MB{E} \SB {\F{s_+ ^3} {6} \rho ^{3/2}}
+\MB{E} \SB {\F{s_+ ^4} {24} \rho ^2}
+\sum_{k=5}^\infty \MB{E} \SB {\F{1} {k!} s_+ ^k \rho ^{k/2}} \NR
%
\NC \leq \NC 1
+s_+ \MB{E} \SB {\R {\rho}}
+\F{s_+ ^2} {2} \MB{E} \SB {\rho}
+\F{s_+ ^3} {6} \MB{E} \SB {\rho ^{3/2}}
+\F{s_+ ^4} {24} \MB{E} \SB {\rho ^2}
+\sum_{k=5}^\infty s_+ ^k \NR[+]
}
Back to the Chernoff Inequality, by taking log on both sides we have
\Disp {
\NC \log p
=\NC \Min {s_+ >0} \log \MB{P} \SB {\rho \geq \RB {1 +\e} \MB{E} \SB {\rho}} \NR
\NC \leq \NC \Min {s_+ >0}
\SB {
  -s_+ \R {1 +\e} +\log \MB{E} \SB {\Ss {e} ^{s_+ \R {\rho}}}
}
\NR[+]
}
Recall, by their Maclaurin series and nature of alternating series, there is the bound
\Disp{
\NC \R {1+\e}
\geq \NC 1 +\F{1}{2} \e -\F{1}{8} \e^2 \NR
\NC \log \RB{1+x}
\leq \NC x -\F{1}{2} x^2 +\F{1}{3} x^3 \NR
}
And, by Cauchy Inequality and our result for \m {\MB{E} \SB {\rho}} and \m {\MB{E} \SB {\rho^2}},
\Disp{
\NC \F{1}{\R{1+N_Y^{-2}}}
\leq \NC \MB{E} \SB {\R{\rho}}
\leq 1, \NR[+]
\NC 1
\leq \NC \MB{E} \SB {\rho ^{3/2}}
\leq \R{1+N_Y^{-2}}, \NR[+]
}
We now try the expansion, and assuming
\Disp{
\NC \NC N_Y^{-2} \ll \e
\ll s \ll 1 \NR[+]
}
If so,
\Disp{
\NC \NC \log p \NR
%
\NC \leq \NC \Min {s_+ >0} \Bigg [
  -\RB {1 +\F{1}{2} \e -\F{1}{8} \e^2} s_+ \NR
  \NC \NC \FourQ +\log \RB{
    1
    +s_+ \MB{E} \SB {\R {\rho}}
    +\F{s_+ ^2} {2} \MB{E} \SB {\rho}
    +\F{s_+ ^3} {6} \MB{E} \SB {\rho ^{3/2}}
    +\F{s_+ ^4} {24} \MB{E} \SB {\rho ^2}
    +\MC {O} \SB {s_+^5}
  }
\Bigg ] \NR
%
\NC =\NC \Min {s_+ >0} \Bigg [
  -\RB {1 +\F{1}{2} \e -\F{1}{8} \e^2} s_+
  +\bigg (
    \MB{E} \SB {\R{\rho}} s_+
    +\F{1}{2} \RB {\MB{E} \SB {\rho} -\MB{E} \SB {\R {\rho}}^2} s_+^2 \NR
  \NC \NC \FourQ +\F{1}{6} \RB {-3 \MB{E} \SB {\R {\rho}} \MB{E} \SB {\rho} +2\MB{E} \SB {\R{\rho}} ^3 +\MB{E} \SB {\rho ^{3/2}}} s_+^3
    +\MC {O} \SB {N_Y^{-6}} +\MC {O} \SB {s_+^4}
  \bigg )
\Bigg ] \NR
%
\NC \leq \NC \Min {s_+ >0} \bigg [
  -\RB {\F{1}{2} \e +\F{1}{8} \e^2} s_+
  +\F{1}{2} \RB {1 -\F{1}{1+N_Y^{-2}}} s_+^2 \NR
  \NC \NC \FourQ +\F{1}{6} \RB {-3 \D \F{1}{\R{1+N_Y^{-2}}} \D 1 +2 +\R{1+N_Y^{-2}}} s_+^3
  +\MC {O} \SB {s_+^4}
\bigg ] \NR
%
\NC =\NC \Min {s_+ >0} \SB{
  -\F{1}{2} \e s_+
  +\F{1}{2} N_Y^{-2} \RB {1 -N_Y^{-2}} s_+^2
  +N_Y^{-2} \RB {\F{1}{3} -\F{5}{24} N_Y^{-2}} s_+^3
} \NR[+]
}
The \m {s_+^3} term has \m {N_Y^{-2}} factor, which justifies the decision that we drop it.
\Disp{
\NC \approx \NC \F{1}{2} \Min {s_+ >0} \RB{ -\e s_+ +N_Y^{-2} s_+^2 } \NR
%
\NC =\NC -\F{\e^2 N_Y^2} {8} \NR[+]
}
The minimum occurs at \m {s =N_Y \e /2}.

% % % % % % % % % % % % % % % % % % % % % % % % % % % % % % % %

Similarly, for any \m {s_- >0}.
\Disp {
\NC p
= \NC \MB{P} \SB {\rho \leq \RB {1 -\e} \MB{E} \SB {\rho}} \NR
\NC =\NC \MB{P} \SB {\Ss {e} ^{s_- \R {\rho}} \leq \Ss {e} ^{s_- \R {1 -\e}}} \NR
\NC \leq \NC \Ss {e} ^{s_- \R {1 -\e}} \MB{E} \SB {\Ss {e} ^{-s_- \R {\rho}}} \NR[+]
}
Here
\Disp{
\NC \NC \MB{E} \SB {\Ss {e} ^{-s_- \R {\rho}}} \NR
%
\NC \leq \NC 1
-s_- \MB{E} \SB {\R {\rho}}
+\F{s_- ^2} {2} \MB{E} \SB {\rho}
-\F{s_- ^3} {6} \MB{E} \SB {\rho ^{3/2}}
+\F{s_- ^4} {24} \MB{E} \SB {\rho ^2}
+\sum_{k=5}^\infty \RB {-s_-} ^k \NR[+]
}
And, miraculously, we get analogous expression

\Disp{
\NC \NC \log p \NR
%
\NC \leq \NC \Min {s_- >0} \Bigg [
  \RB {1 -\F{1}{2} \e -\F{1}{8} \e^2} s_- \NR
  \NC \NC \FourQ +\log \RB{
    1
    -s_- \MB{E} \SB {\R {\rho}}
    +\F{s_- ^2} {2} \MB{E} \SB {\rho}
    -\F{s_- ^3} {6} \MB{E} \SB {\rho ^{3/2}}
    +\F{s_- ^4} {24} \MB{E} \SB {\rho ^2}
    +\MC {O} \SB {s_-^5}
  }
\Bigg ] \NR
%
\NC =\NC \Min {s_- >0} \Bigg [
  \RB {1 -\F{1}{2} \e -\F{1}{8} \e^2} s_-
  +\bigg (
    -\MB{E} \SB {\R{\rho}} s_-
    +\F{1}{2} \RB {\MB{E} \SB {\rho} -\MB{E} \SB {\R {\rho}}^2} s_-^2 \NR
    \NC \NC \FourQ +\F{1}{6} \RB {3 \MB{E} \SB {\R {\rho}} \MB{E} \SB {\rho} -2\MB{E} \SB {\R{\rho}}^3 -\MB{E} \SB {\rho ^{3/2}}} s_-^3
    +\MC {O} \SB {s_-^4}
  \bigg )
\Bigg ] \NR
%
\NC \leq \NC \Min {s_- >0} \bigg [
  \RB {1 -\F{1}{2} \e +\F{1}{8} \e^2 -\F{1}{\R{1+N_Y^{-2}}}} s_-
  +\F{1}{2} \RB {1 -\F{1}{1+N_Y^{-2}}} s_-^2 \NR
  \NC \NC \FourQ +\F{1}{6} \RB {3 \D 1 \D 1 -2 \F {1} {\RB{1+N_Y^{-2}}^{3/2}} -1} s_-^3
  +\MC {O} \SB {s_-^4}
\bigg ] \NR
%
\NC \leq \NC \Min {s_- >0} \bigg [
  \RB {-\F{1}{2} \e +\F{1}{8} \e^2 +\F{1}{2} N_Y^{-2} +\MC {O} \SB {N_Y^{-4}}} s_-
  +\F{1}{2} N_Y^{-2} \RB {1 -N_Y^{-2}} s_-^2 \NR
  \NC \NC \FourQ +N_Y^{-2} \RB {\F{1}{2} -\F{5}{8} N_Y^{-2}} s_-^3
  +\MC {O} \SB {s_-^4}
\bigg ] \NR
%
\NC =\NC \Min {s_- >0} \SB{
  -\F{1}{2} \e s_-
  +\F{1}{2} N_Y^{-2} \RB {1 -N_Y^{-2}} s_-^2
  +N_Y^{-2} \RB {\F{1}{3} -\F{5}{24} N_Y^{-2}} s_-^3
} \NR[+]
%
\NC \approx \NC \F{1}{2} \Min {s_- >0} \RB{ -\e s_- +N_Y^{-2} s_-^2 } \NR
%
\NC =\NC -\F{\e^2 N_Y^2} {8} \NR[+]
}

% % % % % % % % % % % % % % % % % % % % % % % % % % % % % % % %

In summary,

\Result
{Theorem}
{
Let \m {\M{P}} be generated randomly according to ().
Then, for any fixed \m {\V{u} \in \MB {V}_{\MB {C}} \SB {N_Y}}, and for any \m {\e >0},
\Disp{
\NC \MB{P}
\SB {
  \Nm {\VNm {\M{P} \V{u}} _2 ^2 -\VNm {\V{u}} _2 ^2}
  \geq \e \VNm {\V{u}} _2 ^2
}
\leq \NC 2\exp \SB {-\F{N_Y^2}{8} \e^2} \NR[+]
}
}

According to Lemma 5.1 in Baraniuk et.\ al.\ (2008), substitution of relevent quantities yields
\Result
{Theorem}
{
Let the probability that RIP holds for \m {\M{P}} w.r.t.\ \m {\d_s} is \m {p}, then
\Disp{
\NC 1 -p
\leq \NC 2 \D 6^s \D \d_s^{-s} \exp \SB {-\F{N_Y^2 \d_s^2} {32}} \NR[+]
}
}

\stopsubsection
\stopsection


